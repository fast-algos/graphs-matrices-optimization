\documentclass[11pt]{article}
%%%%%%%%% options for the file macros.tex

\def\showauthornotes{1}
\def\showkeys{0}
\def\showdraftbox{0}
% \allowdisplaybreaks[1]

%% Shamelessly adapted from a scribe template by Sanjeev Arora

%%%%%%%%%%%%%% Packages
% \usepackage[active,tightpage]{preview}
% \renewcommand{\PreviewBorder}{1in}
\usepackage[hidelinks]{hyperref}
\usepackage{amsmath,amssymb,amsthm,amstext,amsfonts,bbm,algorithm,algorithmicx,xspace,nicefrac,
  algpseudocode}
\usepackage{color,stmaryrd,enumerate,latexsym,bm,amsfonts,
  subfigure,wrapfig,verbatim,tabularx,textcomp}
\usepackage[small]{caption}
\usepackage{comment} 
\usepackage{epsfig} 
\usepackage{latexsym,nicefrac,bbm}
\usepackage{xspace}
\usepackage{color,fancybox,graphicx,url,subfigure}
\usepackage{enumitem, fullpage}
\usepackage{booktabs}
\usepackage{commath}
\usepackage{mdframed}
\usepackage{pdfsync}
\usepackage{tikz}
\usetikzlibrary {positioning}

%%%%%%%%%%%%%% Use for definitions
\newcommand{\defeq}{\stackrel{\textup{def}}{=}}

\DeclareMathOperator*{\argmax}{arg\,max}
\DeclareMathOperator*{\argmin}{arg\,min}

%%%%%%%%%%%%%% Theorem Environments
\newtheorem{theorem}{Theorem}[section]
\newtheorem{problem}[theorem]{Problem}
\newtheorem{lemma}[theorem]{Lemma}
\newtheorem{definition}[theorem]{Definition}
\newtheorem{corollary}[theorem]{Corollary}
\newtheorem{conjecture}[theorem]{Conjecture}
\newtheorem{proposition}[theorem]{Proposition}
\newtheorem{fact}[theorem]{Fact}
\newtheorem{remark}[theorem]{Remark}

%%%%%%%%%%%%%% Probability stuff
\DeclareMathOperator*{\pr}{\bf Pr}
\DeclareMathOperator*{\av}{\mathbbm{E}}
\DeclareMathOperator*{\var}{\bf Var}

%%%%%%%%%%%%%% Matrix stuff
\newcommand{\tr}[1]{\mathop{\mbox{Tr}}\left({#1}\right)}
\newcommand{\diag}[1]{{\bf Diag}\left({#1}\right)}

%% Notation for integers, natural numbers, reals, fractions, sets, cardinalities
%%and so on
\newcommand{\nfrac}[2]{\nicefrac{#1}{#2}}
\def\abs#1{\left| #1 \right|}
\renewcommand{\norm}[1]{\ensuremath{\left\lVert #1 \right\rVert}}

\newcommand{\floor}[1]{\left\lfloor\, {#1}\,\right\rfloor}
\newcommand{\ceil}[1]{\left\lceil\, {#1}\,\right\rceil}

\newcommand{\pair}[1]{\left\langle{#1}\right\rangle} %for inner product

\newcommand\B{\{0,1\}}      % boolean alphabet  use in math mode
\newcommand\bz{\mathbb Z}
\newcommand\nat{\mathbb N}
\newcommand\rea{\mathbb R}
\newcommand\com{\mathbb{C}}
\newcommand\plusminus{\{\pm 1\}}
\newcommand\Bs{\{0,1\}^*}   % B star use in math mode
\newcommand{\ones}{\mathbbm{1}}
\newcommand{\eye}{\mathbbm{I}}



\newcommand{\V}[1]{\mathbf{#1}\ignorespaces}
\renewcommand\AA{\boldsymbol{\mathit{A}}}
\newcommand\LL{\boldsymbol{\mathit{L}}}

% Used to denote bold commands
                                % e.g. vectors, matrices
\DeclareRobustCommand{\fracp}[2]{{#1 \overwithdelims()#2}}
\DeclareRobustCommand{\fracb}[2]{{#1 \overwithdelims[]#2}}
\newcommand{\marginlabel}[1]%
{\mbox{}\marginpar{\it{\raggedleft\hspace{0pt}#1}}}
\newcommand\card[1]{\left| #1 \right|} %cardinality of set S; usage \card{S}
\renewcommand\set[1]{\left\{#1\right\}} %usage \set{1,2,3,,}
\renewcommand\complement{\ensuremath{\mathsf{c}}}
\newcommand\poly{\mbox{poly}}  %usage \poly(n)
\newcommand{\comp}[1]{\overline{#1}}
\newcommand{\smallpair}[1]{\langle{#1}\rangle}
\newcommand{\ol}[1]{\ensuremath{\overline{#1}}\xspace}
\newcommand{\eps}{\epsilon}
\DeclareMathOperator{\vol}{\mathsf{vol}}


%%%%%%%%%%%%%% Mathcal shortcuts
\newcommand\calF{\mathcal{F}}
\newcommand\calP{\mathcal{P}}
\newcommand\calS{\mathcal{S}}
\newcommand\calG{\mathcal{G}}
\newcommand\calH{\mathcal{H}}
\newcommand\calC{\mathcal{C}}
\newcommand\calD{\mathcal{D}}
\newcommand\calI{\mathcal{I}}
\newcommand\calV{\mathcal{V}}
\newcommand\calK{\mathcal{K}}
\newcommand\calN{\mathcal{N}}
\newcommand\calX{\mathcal{X}}
\newcommand\calU{\mathcal{U}}
\newcommand\calE{\mathcal{E}}
\newcommand\calL{\mathcal{L}}
\newcommand\calR{\mathcal{R}}


%%%%%%%%%%%%%% {{{ authornotes }}}
\definecolor{Mygray}{gray}{0.8}

 \ifcsname ifcommentflag\endcsname\else
  \expandafter\let\csname ifcommentflag\expandafter\endcsname
                  \csname iffalse\endcsname
\fi

\ifnum\showauthornotes=1
\newcommand{\todo}[1]{\colorbox{Mygray}{\color{red}#1}}
\else
\newcommand{\todo}[1]{#1}
\fi

\ifnum\showauthornotes=1
\newcommand{\Authornote}[2]{{\sf\small\color{red}{[#1: #2]}}}
\newcommand{\Authoredit}[2]{{\sf\small\color{red}{[#1]}\color{blue}{#2}}}
\newcommand{\Authorcomment}[2]{{\sf \small\color{gray}{[#1: #2]}}}
\newcommand{\Authorfnote}[2]{\footnote{\color{red}{#1: #2}}}
\newcommand{\Authorfixme}[1]{\Authornote{#1}{\textbf{??}}}
\newcommand{\Authormarginmark}[1]{\marginpar{\textcolor{red}{\fbox{%\Large
#1:!}}}}
\else
\newcommand{\Authornote}[2]{}
\newcommand{\Authoredit}[2]{}
\newcommand{\Authorcomment}[2]{}
\newcommand{\Authorfnote}[2]{}
\newcommand{\Authorfixme}[1]{}
\newcommand{\Authormarginmark}[1]{}
\fi


%%%%%%%%%%%%%% Logical operators
\newcommand\true{\mbox{\sc True}}
\newcommand\false{\mbox{\sc False}}
\def\scand{\mbox{\sc and}}
\def\scor{\mbox{\sc or}}
\def\scnot{\mbox{\sc not}}
\def\scyes{\mbox{\sc yes}}
\def\scno{\mbox{\sc no}}

%% Parantheses
\newcommand{\paren}[1]{\unskip\left({#1}\right)}
\newcommand{\sqparen}[1]{\unskip\left[{#1}\right]}
\newcommand{\curlyparen}[1]{\unskip\left\{{#1}\right\}}
\newcommand{\smallparen}[1]{\unskip({#1})}
\newcommand{\smallsqparen}[1]{\unskip[{#1}]}
\newcommand{\smallcurlyparen}[1]{\unskip\{{#1}\}}

%% short-hands for relational simbols

\newcommand{\from}{:}
\newcommand\xor{\oplus}
\newcommand\bigxor{\bigoplus}
\newcommand{\logred}{\leq_{\log}}
\def\iff{\Leftrightarrow}
\def\implies{\Rightarrow}

%--------------------------------------------------------------------------------------------------------------------------------
% Optimization macros
%--------------------------------------------------------------------------------------------------------------------------------
%\providecommand{\argmax}{\mathop\mathrm{arg max}} % Defining math symbols
%\providecommand{\argmin}{\mathop\mathrm{arg min}}
\providecommand{\arccos}{\mathop\mathrm{arccos}}
\providecommand{\dom}{\mathop\mathrm{dom}}
\providecommand{\diag}{\mathop\mathrm{diag}}
\providecommand{\tr}{\mathop\mathrm{tr}}
%\providecommand{\abs}{\mathop\mathrm{abs}}
\providecommand{\card}{\mathop\mathrm{card}}
\providecommand{\sign}{\mathop\mathrm{sign}}
\providecommand{\conv}{\mathop\mathrm{conv}} % Convex hull
\def\rank#1{\mathrm{rank}({#1})}
\def\supp#1{\mathrm{supp}({#1})}

\providecommand{\minimize}{\mathop\mathrm{minimize}}
\providecommand{\maximize}{\mathop\mathrm{maximize}}
\providecommand{\subjectto}{\mathop\mathrm{subject\;to}}

%\renewcommand\eqref[1]{Eq.~(\ref{#1})}

\def\openright#1#2{\left[{#1}, {#2}\right)}

%--------------------------------------------------------------------------------------------------------------------------------
% Vectors and matrices
%--------------------------------------------------------------------------------------------------------------------------------
\newcommand{\boldone}{\mbf{1}} % Bold 1
\newcommand{\ident}{\mbf{I}} % Identity matrix
% \def\v#1{\mbi{#1}} % Vector notation
%\def\norm#1{\left\|{#1}\right\|} % A norm with 1 argument
\newcommand{\onenorm}[1]{\norm{#1}_1} % L1 norm
\newcommand{\twonorm}[1]{\norm{#1}_2} % L2 norm
\newcommand{\infnorm}[1]{\norm{#1}_{\infty}} % Linfty norm
\newcommand{\opnorm}[1]{\norm{#1}_{\text{op}}} % Operator norm
\newcommand{\fronorm}[1]{\norm{#1}_{\text{F}}} % Frobenius norm
\newcommand{\nucnorm}[1]{\norm{#1}_{*}} % Frobenius norm
\def\staticnorm#1{\|{#1}\|} % A static norm that does not resize with input
\newcommand{\statictwonorm}[1]{\staticnorm{#1}_2} % L2 norm
\newcommand{\inner}[1]{{\langle #1 \rangle}} % inner product
\newcommand{\binner}[2]{\left\langle{#1},{#2}\right\rangle} % Inner product with expandable brackets
\def\what#1{\widehat{#1}}

\def\twovec#1#2{\left[\begin{array}{c}{#1} \\ {#2}\end{array}\right]}
\def\threevec#1#2#3{\left[\begin{array}{c}{#1} \\ {#2} \\ {#3} \end{array}\right]}
\def\nvec#1#2#3{\left[\begin{array}{c}{#1} \\ {#2} \\ \vdots \\ {#3}\end{array}\right]} % An n-vector with three arguments
\DeclareMathOperator\spn{span}


%% macros to write pseudo-code

\newlength{\pgmtab}  %  \pgmtab is the width of each tab in the
\setlength{\pgmtab}{1em}  %  program environment
 \newenvironment{program}{\renewcommand{\baselinestretch}{1}%
\begin{tabbing}\hspace{0em}\=\hspace{0em}\=%
\hspace{\pgmtab}\=\hspace{\pgmtab}\=\hspace{\pgmtab}\=\hspace{\pgmtab}\=%
\hspace{\pgmtab}\=\hspace{\pgmtab}\=\hspace{\pgmtab}\=\hspace{\pgmtab}\=%
\+\+\kill}{\end{tabbing}\renewcommand{\baselinestretch}{\intl}}
\newcommand {\BEGIN}{{\bf begin\ }}
\newcommand {\ELSE}{{\bf else\ }}
\newcommand {\IF}{{\bf if\ }}
\newcommand {\FOR}{{\bf for\ }}
\newcommand {\TO}{{\bf to\ }}
\newcommand {\DO}{{\bf do\ }}
\newcommand {\WHILE}{{\bf while\ }}
\newcommand {\ACCEPT}{{\bf accept}}
\newcommand {\REJECT}{\mbox{\bf reject}}
\newcommand {\THEN}{\mbox{\bf then\ }}
\newcommand {\END}{{\bf end}}
\newcommand {\RETURN}{\mbox{\bf return\ }}
\newcommand {\HALT}{\mbox{\bf halt}}
\newcommand {\REPEAT}{\mbox{\bf repeat\ }}
\newcommand {\UNTIL}{\mbox{\bf until\ }}
\newcommand {\TRUE}{\mbox{\bf true\ }}
\newcommand {\FALSE}{\mbox{\bf false\ }}
\newcommand {\FORALL}{\mbox{\bf for all\ }}
\newcommand {\DOWNTO}{\mbox{\bf down to\ }}

% Theorem-type environments
% \theoremstyle{break} 
% \theoremheaderfont{\scshape}
% \theorembodyfont{\slshape}
% \newtheorem{Thm}{Theorem}[section]
% \newtheorem{Lem}[Thm]{Lemma}
% \newtheorem{Cor}[Thm]{Corollary}
% \newtheorem{Prop}[Thm]{Proposition}
% % \theoremstyle{plain} 
% % \theorembodyfont{\rmfamily} 
% \newtheorem{Ex}[Thm]{Exercise}
% \newtheorem{Exa}[Thm]{Example}
% \newtheorem{Rem}[Thm]{Remark}
% % \theorembodyfont{\itshape}
% \newtheorem{Def}[Thm]{Definition}
% \newtheorem{Conj}[Thm]{Conjecture}
% \newtheorem{Obs}[Thm]{Observation}
% \newtheorem{Ques}[Thm]{Question}
%\newenvironment{proof}{\noindent {\sc Proof:}}{$\Box$ \medskip} 
\newenvironment{problems} % Definition of problems
 {\renewcommand{\labelenumi}{\S\theenumi}
	\begin{enumerate}}{\end{enumerate}}


%%%%%%%%%%%%%%%%% Proof Environments

\def\FullBox{\hbox{\vrule width 6pt height 6pt depth 0pt}}
%
%\def\qed{\ifmmode\qquad\FullBox\else{\unskip\nobreak\hfil
%\penalty50\hskip1em\null\nobreak\hfil\FullBox
%\parfillskip=0pt\finalhyphendemerits=0\endgraf}\fi}

\def\qedsketch{\ifmmode\Box\else{\unskip\nobreak\hfil
\penalty50\hskip1em\null\nobreak\hfil$\Box$
\parfillskip=0pt\finalhyphendemerits=0\endgraf}\fi}

%\newenvironment{proof}{\begin{trivlist} \item {\bf Proof:~~}}
 %  {\qed\end{trivlist}}

\newenvironment{proofsketch}{\begin{trivlist} \item {\bf
Proof Sketch:~~}}
  {\qedsketch\end{trivlist}}

\newenvironment{proofof}[1]{\begin{trivlist} \item {\bf Proof
#1:~~}}
  {\qed\end{trivlist}}

\newenvironment{claimproof}{\begin{quotation} \noindent
{\bf Proof of claim:~~}}{\qedsketch\end{quotation}}


%%%%%%%%%%%%%%%%%%%%%%%%%%%%%%%%%%%%%%%%%%%%%%%%%%%%%%%%%%%%%%%%%%%%%%%%%%%
%%%%%%%%%%%%%%%%%%%%%%%%%%%%%%%%%%%%%%%%%%%%%%%%%%%%%%%%%%%%%%%%%%%%%%%%%%%




\newlength{\tpush}
\setlength{\tpush}{2\headheight}
\addtolength{\tpush}{\headsep}

\newcommand{\handout}[5]{
   \noindent
   \begin{center}
   \framebox{ \vbox{ \hbox to \textwidth { {\bf \coursenum\ :\  \coursename} \hfill #5 }
       \vspace{3mm}
       \hbox to \textwidth { {\Large \hfill #2  \hfill} }
       \vspace{1mm}
       \hbox to \textwidth { {\it #3 \hfill #4} }
     }
   }
   \end{center}
   \vspace*{4mm}
   \newcommand{\lecturenum}{#1}
   \addcontentsline{toc}{chapter}{Lecture #1 -- #2}
}

\newcommand{\lecturetitle}[4]{\handout{#1}{#2}{Lecturer: \courseprof
  }{Scribes: #3}{Date: #4}}
\newcommand{\guestlecturetitle}[5]{\handout{#1}{#2}{Lecturer:
    #4}{Scribe: #3}{Lecture #1 - #5}}


%%%%%%%%%%%%%%%%%%%%%%%%%%%%%%%%%%%%%%%%%%%%%%%%%%%%%%%%%
%%% Commands to include figures


%% PSfigure

\newcommand{\PSfigure}[3]{\begin{figure}[t] 
  \centerline{\vbox to #2 {\vfil \psfig{figure=#1.eps,height=#2} }} 
  \caption{#3}\label{#1} 
  \end{figure}} 
\newcommand{\twoPSfigures}[5]{\begin{figure*}[t]
  \centerline{%
    \hfil
    \begin{tabular}{c}
        \vbox to #3 {\vfil\psfig{figure=#1.eps,height=#3}} \\ (a)
    \end{tabular}
    \hfil\hfil\hfil
    \begin{tabular}{c}
        \vbox to #3 {\vfil\psfig{figure=#2.eps,height=#3}} \\ (b)
    \end{tabular}
    \hfil}
  \caption{#4}
  \label{#5}
%  \sublabel{#1}{(a)}
%  \sublabel{#2}{(b)}
  \end{figure*}}

\newcounter{fignum}

% fig
%command to insert figure. usage \fig{name}{h}{caption}
%where name.eps is the postscript file and h is the height in inches
%The figure is can be referred to using \ref{name}
\newcommand{\fig}[3]{%
\begin{minipage}{\textwidth}
\centering\epsfig{file=#1.eps,height=#2}
\caption{#3} \label{#1}
\end{minipage}
}%


% ffigure
% Usage: \ffigure{name of file}{height}{caption}{label}
\newcommand{\ffigure}[4]{\begin{figure} 
  \centerline{\vbox to #2 {\hfil \psfig{figure=#1.eps,height=#2} }} 
  \caption{#3}\label{#4} 
  \end{figure}} 

% ffigureh
% Usage: \ffigureh{name of file}{height}{caption}{label}
\newcommand{\ffigureh}[4]{\begin{figure}[!h] 
  \centerline{\vbox to #2 {\vfil \psfig{figure=#1.eps,height=#2} }} 
  \caption{#3}\label{#4} 
  \end{figure}} 


% {{{ draftbox }}}
\ifnum\showdraftbox=1
\newcommand{\draftbox}{\begin{center}
  \fbox{%
    \begin{minipage}{2in}%
      \begin{center}%
%        \begin{Large}%
          \large\textsc{Working Draft}\\%
%        \end{Large}\\
        Please do not distribute%
      \end{center}%
    \end{minipage}%
  }%
\end{center}
\vspace{0.2cm}}
\else
\newcommand{\draftbox}{}
\fi


%% Complexity classes
\newcommand\p{\mbox{\bf P}\xspace}
\newcommand\np{\mbox{\bf NP}\xspace}
\newcommand\cnp{\mbox{\bf coNP}\xspace}
\newcommand\sigmatwo{\mbox{\bf $\Sigma_2$}\xspace}
\newcommand\ppoly{\mbox{\bf $\p_{\bf /poly}$}\xspace}
\newcommand\sigmathree{\mbox{\bf $\Sigma_3$}\xspace}
\newcommand\pitwo{\mbox{\bf $\Pi_2$}\xspace}
\newcommand\rp{\mbox{\bf RP}\xspace}
\newcommand\zpp{\mbox{\bf ZPP}\xspace}
\newcommand\bpp{\mbox{\bf BPP}\xspace}
\newcommand\ph{\mbox{\bf PH}\xspace}
\newcommand\pspace{\mbox{\bf PSPACE}\xspace}
\newcommand\npspace{\mbox{\bf NPSPACE}\xspace}
\newcommand\dl{\mbox{\bf L}\xspace}
\newcommand\ma{\mbox{\bf MA}\xspace}
\newcommand\am{\mbox{\bf AM}\xspace}
\newcommand\nl{\mbox{\bf NL}\xspace}
\newcommand\conl{\mbox{\bf coNL}\xspace}
\newcommand\sharpp{\mbox{\#{\bf P}}\xspace}
\newcommand\parityp{\mbox{$\oplus$ {\bf P}}\xspace}
\newcommand\ip{\mbox{\bf IP}\xspace}
\newcommand\pcp{\mbox{\bf PCP}}
\newcommand\dtime{\mbox{\bf DTIME}}
\newcommand\ntime{\mbox{\bf NTIME}}
\newcommand\dspace{\mbox{\bf SPACE}\xspace}
\newcommand\nspace{\mbox{\bf NSPACE}\xspace}
\newcommand\cnspace{\mbox{\bf coNSPACE}\xspace}
\newcommand\exptime{\mbox{\bf EXP}\xspace}
\newcommand\nexptime{\mbox{\bf NEXP}\xspace}
\newcommand\genclass{\mbox{$\cal C$}\xspace}
\newcommand\cogenclass{\mbox{\bf co$\cal C$}\xspace}
\newcommand\size{\mbox{\bf SIZE}\xspace}
\newcommand\sig{\mathbf \Sigma}
\newcommand\pip{\mathbf \Pi}

%%Computational problems
\newcommand\sat{\mbox{SAT}\xspace}
\newcommand\tsat{\mbox{3SAT}\xspace}
\newcommand\tqbf{\mbox{TQBF}\xspace}
\allowdisplaybreaks

\usepackage{float}
\usepackage{tikz}
\usetikzlibrary{hobby,calc}
\usetikzlibrary{arrows.meta}
\usetikzlibrary{patterns}
\usepackage{pgfplots}

\usepackage{thmtools}
\usepackage{thm-restate}


\usepackage[
    backend=biber,
% giveninits=true,
% natbib=true,
    style=alphabetic,
    url=false, 
 %   doi=true,
    hyperref,
    backref=true,
    backrefstyle=none,
    maxbibnames=10,
    sortcites
]{biblatex}
\addbibresource{refs.bib}

%%%%%%%%% Authornotes
\newcommand{\Snote}{\Authornote{S}}

\newenvironment{tight_enumerate}{
\begin{enumerate}
 \setlength{\itemsep}{2pt}
 \setlength{\parskip}{1pt}
}{\end{enumerate}}
\newenvironment{tight_itemize}{
\begin{itemize}
 \setlength{\itemsep}{2pt}
 \setlength{\parskip}{1pt}
}{\end{itemize}}


\renewcommand\vec{\V}
\newcommand\itxt[1]{\intertext{\indent#1}}

\addbibresource{papers.bib}

\newcommand{\cov}{\mathbf{cov}}
\newcommand{\hit}{\mathbf{hit}}
\newcommand{\infec}{\mathbf{infec}}


\newcommand{\E}{\mathbb E}
\renewcommand{\Pr}{\mathbb P}
\newcommand{\ind}{\mathbb I}

\theoremstyle{remark}
\newtheorem*{sketch}{Proof sketch}

%%%%%%%%%%%%%%%%%%%%%%%%%

\begin{document}

\newcommand{\coursenum}{{CSC 2421H}}
\newcommand{\coursename}{{Graphs, Matrices, and Optimization}}
\newcommand{\courseprof}{Sushant Sachdeva}

\lecturetitle{1}{The Coalescing-Branching Random Walk on Expanders}{Sasha Voitovych}{Dec 22 2022}

\section{The Coalescing-Branching Random Walk}

\subsection{Refresher on random walks}

We first give a refresher on some concepts from 
the study of random walks on graphs. For simplicity, let's consider an undirected unweighted graph.

\begin{definition}
\label{def:random-walk}
    Let $G = (V,E)$ be an undirected unweighted graph. Then random walk started at $s \in V$ is a Markov process $(X_t)_{t \ge 0}$ such that $X_0 = s$ and $X_{t+1} \mid X_t$ follows a uniform distribution over the neighbourhood of $X_t$ in $G$ for every $t \ge 0$.
\end{definition}

Simply speaking, a random walk can be described by movement of a particle, which at each step moves to a uniformly at random chosen neighbour of its current position. Figure~\ref{fig:cobra1} gives a simple example of a random walk moving for three time steps. 

We call the \emph{cover time} $\cov(v)$ of the vertex $v \in V$ the minimum time the random walk visits all vertices in the graph $G$. We call the cover time $\cov(G)$ of the graph $G$ the maximal expected cover time $\cov(v)$ across all $v \in V$. In other words,
\[
\cov(G) = \max_{v \in V} \E\left[\cov(v)\right].
\]
Work of \cite{Aldous1989LowerBF} showed that that cover time of \emph{any} graph must be in $\Omega(n\log(n))$. On the other hand, the cover time of any $(d,\lambda)$ expander with $\lambda < 1$ is in $O(n\log(n))$ (see, e.g., Theorem 5 of~\cite{Broder1988BoundsOT}). In other words, expander graphs give an optimal cover time by a random walk.

\subsection{Introducing $k$-COBRA walk}

We introduce a generalized version of a simple random walk, called \emph{a $k$-coalescing branching random walk}, or $k$-COBRA walk for short.

First, we give a verbal description. Let $k\in \nat$ and consider a graph $G = (V,E)$ and a single particle located at a vertex $v \in V$. At each time step, each particle \emph{branches} into $k$ particles, which travel neighbouring chosen independently and uniformly at random. Then, at the end of the time step, all particles at the same location \emph{coalesce} into one particle. See Figure~\ref{fig:cobra2} for a simple realization of a $2$-COBRA walk after three time steps. 

We give a more formal definition below. For a non-empty set $S$, let $\mathcal U_k(S)$ be a distribution over the subsets of $S$ obtained by picking $k$ elements of $S$ uniformly at random \emph{with replacement}.  

\xdefinecolor{cobra}{cmyk}{0,0.20, 1.0, 0.02}
\xdefinecolor{node}{cmyk}{0.5,0.26,0.0,0.15} 
\xdefinecolor{curious}{cmyk}{0, 0.83, 0.80, 0.0}
\xdefinecolor{hidden}{cmyk}{0, 0, 0, 0.35}

\tikzset{hide labels/.style={every label/.append style={text opacity=0}}}

\begin{figure}
    \centering
    \begin{minipage}{0.5\textwidth}
    \centering
        \begin{tikzpicture}[scale=3, rotate=270, hide labels]%[hide labels]

        \begin{scope}
            \path[draw,use Hobby shortcut,closed=true, fill=cobra, fill opacity=0.15]
            (1.28,-0.05) .. (1.15,0.08) .. (1.02,-0.05) .. (1.15,-0.18);
            \node[circle] (curious) at (1.25,-0.25) {$X_3$};

            \path[draw,use Hobby shortcut,closed=true, fill=cobra, fill opacity=0.15]
            (-0.23,-0.05) ..  (-0.1,-0.18) ..  (0.03,-0.05) ..  (-0.1,0.08);
            \node[circle] (curious) at (-0.06,-0.27) {$X_0$};
        \end{scope}

        

        %(-0.1,-0.05)
    
        \begin{scope}[every node/.style={circle, draw, inner sep=0pt,text width=2mm,fill=cobra}]
            %%% cobra nodes    
            \node[label=below:{\tiny $A$}] (A) at (-0.1,-0.05) {};
            \node[label=below:{\tiny $B$}] (B) at (0.3,-0.25) {};
            \node[label=below:{\tiny $D$}] (D) at (1.15,-0.05) {};

            \node[label=below:{\tiny $2$}] (2) at (0.7,0) {};


        \end{scope}
        % \node[anchor=west] at ([shift=(160:0.5)]10) {$u$};
        \begin{scope}[every node/.style={circle, draw, inner sep=0pt,text width=2mm,fill=node}]
            %%% regular nodes
            % \node[label=below:{\tiny $G$}] (G) at (1.8,0.5) {};
            % \node[label=below:{\tiny $H$}] (H) at (1.9,0.15) {};
            \node[label=below:{\tiny $1$}] (1) at (0.3,0.2) {};
            \node[label=below:{\tiny $5$}] (5) at (0.65,0.4) {};
            \node[label=below:{\tiny $8$}] (8) at (1.15,-0.4) {};
            \node[label=below:{\tiny $9$}] (9) at (1.1,0.3) {};
            \node[label=below:{\tiny $7$}] (7) at (1.0,-0.7) {};
            \node[label=below:{\tiny $14$}] (14) at (0.95,0.7) {};
            \node[label=below:{\tiny $C$}] (C) at (0.7,-0.4) {};

            \node[label=below:{\tiny $3$}] (3) at (-0.05,-0.65) {};
            \node[label=below:{\tiny $4$}] (4) at (0.1,0.55) {};
            \node[label=below:{\tiny $17$}] (17) at (0.3,-0.55) {};

            \node[label=below:{\tiny $6$}] (6) at (0.55,-0.8) {};
            
            \node[label=below:{\tiny $12$}] (12) at (-0.25,0.3) {};
            \node[label=below:{\tiny $18$}] (18) at (-0.2,0.7) {};

            \node[label=below:{\tiny $13$}] (13) at (-0.3,-0.35) {};
            \node[label=below:{\tiny $16$}] (16) at (0.5,0.73) {};

            \node[label=below:{\tiny $E$}] (E) at (1.55,0.2) {};
            \node[label=below:{\tiny $F$}] (F) at (1.5,-0.2) {};
            
            \node[label=below:{\tiny $10$}] (10) at (1.4,0.55) {};
            \node[label=below:{\tiny $11$}] (11) at (1.55,-0.55) {};
            \node[label=below:{\tiny $15$}] (15) at (1.4,-0.85) {};
            % \node (G) at (1.8,0.5) {};
            % \node (H) at (1.9,0.15) {};
            % \node (I) at (2.0,-0.35) {};


            % \node[label=below:{\tiny $I$}] (I) at (2.0,-0.35) {};
        \end{scope}

        
        \begin{scope}[>={Stealth[cobra]},
                      every edge/.style={draw=cobra, thick}]
            %%% cobra edges

            % \path[->] (B) edge (2) edge (C);
            % \path[->] (2) edge (9) edge (D);
            % \path[->] (C) edge (D) edge (8);
            % \path[->] (8) edge (11) edge (F);
            % \path[->] (D) edge (E) edge (F);
            % \path[->] (9) edge (10) edge (14);

            \path[->] (A) edge (B) (B) edge (2) (2) edge (D);

        \end{scope}

        
        
        % \begin{scope}[>={Stealth[curious]},
        %               every edge/.style={draw=curious, thick}]
        %     %% curious arrow 
        %     \path[->,dashed,thick] (curious) edge[in=30,out=100, looseness=1] node [above right, pos=0.25] {$s?$} (10);
        % \end{scope}
    \end{tikzpicture}

    \subcaption{Example of a random walk}
    \label{fig:cobra1}
    \end{minipage}%
    \begin{minipage}{0.5\textwidth}
    \centering
        \begin{tikzpicture}[scale=3, rotate=270, hide labels]%[hide labels]

        \begin{scope}
            \path[draw,use Hobby shortcut,closed=true, fill=cobra, fill opacity=0.15]
            (1.0,-0.9) .. (1.3,-0.05) .. (0.95,0.9) .. (0.9,-0.05);
            \node[circle] (curious) at (1.1, -1.02) {$C_3$};

            \path[draw,use Hobby shortcut,closed=true, fill=cobra, fill opacity=0.15]
            (-0.23,-0.05) ..  (-0.1,-0.18) ..  (0.03,-0.05) ..  (-0.1,0.08);
            \node[circle] (curious) at (-0.06,-0.27) {$C_0$};

        \end{scope}

        

        %(-0.1,-0.05)
    
        \begin{scope}[every node/.style={circle, draw, inner sep=0pt,text width=2mm,fill=cobra}]
            %%% cobra nodes    
            \node[label=below:{\tiny $A$}] (A) at (-0.1,-0.05) {};
            \node[label=below:{\tiny $B$}] (B) at (0.3,-0.25) {};
            \node[label=below:{\tiny $C$}] (C) at (0.7,-0.4) {};
            \node[label=below:{\tiny $D$}] (D) at (1.15,-0.05) {};
            \node[label=below:{\tiny $1$}] (1) at (0.3,0.2) {};
            \node[label=below:{\tiny $5$}] (5) at (0.65,0.4) {};

            \node[label=below:{\tiny $2$}] (2) at (0.7,0) {};
            \node[label=below:{\tiny $8$}] (8) at (1.15,-0.4) {};
            \node[label=below:{\tiny $9$}] (9) at (1.1,0.3) {};
            \node[label=below:{\tiny $7$}] (7) at (1.0,-0.7) {};
            \node[label=below:{\tiny $14$}] (14) at (0.95,0.7) {};


        \end{scope}
        % \node[anchor=west] at ([shift=(160:0.5)]10) {$u$};
        \begin{scope}[every node/.style={circle, draw, inner sep=0pt,text width=2mm,fill=node}]
            %%% regular nodes
            % \node[label=below:{\tiny $G$}] (G) at (1.8,0.5) {};
            % \node[label=below:{\tiny $H$}] (H) at (1.9,0.15) {};

            \node[label=below:{\tiny $3$}] (3) at (-0.05,-0.65) {};
            \node[label=below:{\tiny $4$}] (4) at (0.1,0.55) {};
            \node[label=below:{\tiny $17$}] (17) at (0.3,-0.55) {};

            \node[label=below:{\tiny $6$}] (6) at (0.55,-0.8) {};
            
            \node[label=below:{\tiny $12$}] (12) at (-0.25,0.3) {};
            \node[label=below:{\tiny $18$}] (18) at (-0.2,0.7) {};

            \node[label=below:{\tiny $13$}] (13) at (-0.3,-0.35) {};
            \node[label=below:{\tiny $16$}] (16) at (0.5,0.73) {};

            \node[label=below:{\tiny $E$}] (E) at (1.55,0.2) {};
            \node[label=below:{\tiny $F$}] (F) at (1.5,-0.2) {};
            
            \node[label=below:{\tiny $10$}] (10) at (1.4,0.55) {};
            \node[label=below:{\tiny $11$}] (11) at (1.55,-0.55) {};
            \node[label=below:{\tiny $15$}] (15) at (1.4,-0.85) {};
            % \node (G) at (1.8,0.5) {};
            % \node (H) at (1.9,0.15) {};
            % \node (I) at (2.0,-0.35) {};


            % \node[label=below:{\tiny $I$}] (I) at (2.0,-0.35) {};
        \end{scope}

        
        \begin{scope}[>={Stealth[cobra]},
                      every edge/.style={draw=cobra, thick}]
            %%% cobra edges

            % \path[->] (B) edge (2) edge (C);
            % \path[->] (2) edge (9) edge (D);
            % \path[->] (C) edge (D) edge (8);
            % \path[->] (8) edge (11) edge (F);
            % \path[->] (D) edge (E) edge (F);
            % \path[->] (9) edge (10) edge (14);

            \path[->] (A) edge (1) edge (B);
            \path[->] (1) edge (5) edge (2);
            \path[->] (B) edge (2) edge (C);
            \path[->] (5) edge (14) edge (9);
            \path[->] (2) edge (D) edge (9);
            \path[->] (C) edge (7) edge (8);

        \end{scope}

        
        
        % \begin{scope}[>={Stealth[curious]},
        %               every edge/.style={draw=curious, thick}]
        %     %% curious arrow 
        %     \path[->,dashed,thick] (curious) edge[in=30,out=100, looseness=1] node [above right, pos=0.25] {$s?$} (10);
        % \end{scope}
    \end{tikzpicture}

    \subcaption{Example of a $2$-COBRA walk}
    \label{fig:cobra2}
    \end{minipage}

    
    \caption{A comparison between a simple random walk and a $2$-COBRA walk. Figure~\ref{fig:cobra1} and figure~\ref{fig:cobra2} show a realization of a random walk and a $2$-COBRA walk respectively after three time steps. Here, $X_t$ and $C_t$ denote position of a random walk and an active set of a COBRA walk respectively at time $t$. }
    \label{fig:cobra}
\end{figure}

\begin{definition}
\label{def:cobra}
    Let $k\in \nat$ and let $G = (V,E)$ be an undirected unweighted graph. 
    Then a $k$-COBRA walk started at $v \in V$ is a Markov process $(C_t)_{t\ge0}$ such that $C_0 = \{v\}$ and the distribution $C_{t+1} \mid C_t$ can be described as follows. For for all $u \in C_t$, consider independent random variables $B(u) \sim \mathcal U_k(N(u))$. Then $C_{t+1} \mid C_t$ is the same in law as $\cup_{u \in C_t} B(u)$.
\end{definition}

We will call $C_t$ \emph{the active set} of the COBRA walk at time $t$. Note that for $k = 1$, $k$-COBRA walk is just a simple random walk. For simplicity of exposition, we will focus in the rest of the report on analyzing a $2$-COBRA walk. We aim to establish the following theorem about the covering time of $2$-COBRA walk on expanders\footnote{In fact, result for $2$-COBRA walk also implies the result for $k$-COBRA walk with $k \ge 3$.}.
\begin{theorem}
    \label{thm:main}
    Let $G$ be a $(d,\lambda)$-expander with $n-1 \ge d \ge 3$ and constant $\lambda < 1$. Then covering time of the $2$-COBRA walk on $G$ is in $O(\log(n))$ in expectation and with high probability. 
\end{theorem}
In this report, ``with high probability'' means with probability at least $1 - O\left(1/n\right)$. In other words, $2$-COBRA walk has an \emph{exponential speed up} in covering time compared to a simple random walk (a.k.a., $1$-COBRA walk), which has covering time in $\Omega(n\log(n))$.

% \subsection{Motivation}

% \subsection{Heuristics for covering by a COBRA walk}

\section{Dual BIPS process}

Covering time of a COBRA walk is challenging to analyze directly, for which we give a heuristic explanation. Consider the evolution of a $2$-COBRA walk on the graph $G = (V,E)$, and let us look at the moment when the size of the active set $C_t$ is roughly half of the size of $V$. Then the probability that any two particles coalesce in the following round is also roughly $1/2$. Then the size of the next active set $C_{t+1}$ will be roughly $1/2\cdot 2|C_t|$ (where factor $2$ comes from every particle branching, and $1/2$ comes from the probability of coalescing). In other words, after time $t$ the size of the active set \emph{doesn't grow}, but just shuffles through $V$. 

It's very difficult to analyze how exactly this set shuffles through $V$ due to the collisions. We circumvent this by considering an alternative process on $G$ called Biased Infection with Persistent Source (or BIPS), which relates to a COBRA walk, but does not not have the ``shuffling'' phase. 

\subsection{Biased Infection with Persistent Source}

We first give a verbal description of $k$-BIPS process. Consider graph $G = (V,E)$ and a single infected node $v \in V$. We will call $v$ the source of the infection, and $v$ will stay permanently infected. At every time step, every node (except $v$) selects $k$ of its neighbours uniformly at random with replacement. If one of its neighbours is infected, the node becomes infected. It can be defined more formally in the following way.

\begin{definition}
\label{def:bips}
    Let $k\in \nat$ and let $G = (V,E)$ be an undirected unweighted graph. 
    Then a $k$-BIPS walk started at $v \in V$ is a Markov process $(A_t)_{t\ge0}$ such that $A_0 = \{v\}$ and the distribution $A_{t+1} \mid A_t$ can be described as follows. For all $u \in V$, consider independent random variables $B'(u) \sim \mathcal U_k(N(u))$. Then $A_{t+1} \mid A_t$ is the same in law as $\{v\} \cup \left\{u \in V: A_t \cap B'(u)\not = \emptyset \right\}$.
 \end{definition}

We note that if all nodes become infected (i.e., $A_t = V$), then the graph will stay infected (i.e., $A_{t+1} = V$). We will show that not only this happens with probability $1$, but also that it happens quite fast, e.g., in $O(\log(n))$ time.

\subsection{Duality with COBRA walk}

The hitting times of a COBRA walk can be upper bounded using the BIPS process. We will denote $\hit(C,v)$ to be the \emph{hitting time} of a vertex $v$ by a $2$-COBRA walk started from an active set $C_0 = C$. The following result relates hitting times of a COBRA walk to infection times of the BIPS process.

\begin{lemma}
\label{lemma:dual}
    Let $C \subseteq V$ and $v \in V$. Let $A_t$ be the infected set of $2$-BIPS process. For every $t \ge 0$
    \[\Pr\left[\hit(C,v) > t \right] = \Pr\left[A_t \cap C = \emptyset \mid A_0 = \{v\} \right]. \]
\end{lemma}
\begin{proof}
    We will prove the statement by induction. For $t = 0$, both probabilities are $1$ if $v \in C$ and both are $0$ if $v \not \in C$. This establishes the induction base. 

    Suppose the lemma statement holds for $t$. We will show it also holds for $t + 1$. First, consider $\Pr\left[A_{t+1} \cap C = \emptyset\right]$. Consider the $k$-BIPS process at time step $t + 1$ and consider $k$ random neighbours each node from $C$ selects. For $u \in C$, let $B'(u) \sim \mathcal U_k(N(u))$ be such random subset of neighbours. Node $u$ does not become infected at step $t+1$ if and only if $B'(u) \cap A_t = \emptyset$ (i.e., node $u$ didn't sample an infected node among it's neighbours). Denote $B'(C) = \cup_{u \in C} B'(u)$.
    Then none of the nodes in $C$ are infected at round $t+1$ iff $B'(C) \cap A_t = \emptyset$. In other words,
    \begin{equation}
    \label{eq:dual-bips}
    \Pr\left[A_{t+1} \cap C = \emptyset\right] =  \sum_{B \subseteq V} \Pr\left[A_{t} \cap B = \emptyset\right] \Pr[B'(C) = B]= \sum_{B\subseteq V} \Pr[\hit(B,v) > t] \Pr[B'(C) = B],
    \end{equation}
    where second transition is by induction hypothesis.
    On the other hand, consider the COBRA walk starting with $C_0 = C$ at step $0$. For $u\in C$, let $B(u) \sim \mathcal C_k(u)$ to be the random set of neighbours to which particles travel from $u$ in the next round. Then $B(C) = \cup_{u \in C} B(u)$ will be the active set of the COBRA walk at time $1$. Then
    \begin{equation}
    \label{eq:dual-cobra}
    \Pr\left[\hit(C, v) > t + 1\right] = \sum_{B \subseteq V}  \Pr\left[\hit(B,v) > t\right] \Pr[B(C) = B].
    \end{equation}
    On the other hand, $B'(u)$ is the same in law as $B(u)$. Then $B'(C) = \cup_{u \in C} B'(u)$ is the same in law as $B(C) = \cup_{u \in C} B(u)$. Then, for any $B$, we have $\Pr[B(C) = B] = \Pr[B'(C) = B]$. Then, from (\ref{eq:dual-bips}) and (\ref{eq:dual-cobra}) we may conclude 
    \[\Pr\left[A_t \cap C = \emptyset \mid A_0 = \{v\} \right] = \Pr\left[\hit(v) > t \mid C_0 = C \right].\]
\end{proof}

Let $\infec(v)$ be the time it takes for the $2$-BIPS process to infect the whole graph. Letting $C = \{u\}$ in Lemma~\ref{lemma:dual}, we obtain \begin{equation}
\label{eq:infec}
\Pr[\infec(v) > t] \ge \Pr\left[u \in A_t \mid A_0 = \{v\} \right] = \Pr\left[\hit(u,v) > t \right].
\end{equation}
In the rest of the report we will aim to show that $\infec(v) \in O(\log(n))$ with probability at least $1 - O\left(\frac{1}{n^3}\right)$. This will imply that for a fixed $u$, $\cov(u) \in O(\log(n))$ with probability $1 - O\left(\frac{1}{n^2}\right)$, which also implies the result in expectation\footnote{Here and in the rest of the report $\cov(u)$ refers to the cover time of a $2$-COBRA walk started from vertex $v$}.

\section{Covering time of $2$-BIPS process}

We now give a result for the expected growth rate of the $2$-BIPS process, which is essential to the rest of the proof. Unlike $2$-COBRA walk, infected set of the $2$-BIPS set grows (in expectation) at every step. 

\begin{lemma}
    \label{lemma:bips-growth}
    Let $A_t$ be the infected set of the $2$-BIPS process at step $t$. Then for any $A \subseteq V$
    \[
    \E\left[|A_{t+1}| \mid A_t = A\right] \ge |A|\left(1 + (1-\lambda^2)\left(1- \frac{|A|}{n}\right)\right).
    \]
\end{lemma}
\begin{proof}
    Suppose $A_t = A$. Let $\ind(u \in A_{t+1})$ be the indicator random variable of the event $u \in A_{t+1}$
    \[
    |A_{t+1}| = \sum_{u \in V} \ind(u \in A_{t+1}).
    \]
    Note that $\ind(v \in A_{t+1})$ is always $1$, and for $u \not = v$ we have
    \[
    \E[\ind(u \in A_{t+1}) \mid A_t = A] = \left(1 - \left(1 - \frac{\deg_A(u)}{d}\right)^2\right) = \frac{2\deg_A(u)}{d} - \frac{\deg_A(u)^2}{d^2}.
    \] 
    Therefore,
    \begin{align*}
    \E[|A_{t+1}| \mid A_{t} = A ] &= 1 
 + \sum_{u \in V\setminus \{v\}} \left(\frac{2\deg_A(u)}{d} - \frac{\deg_A(u)^2}{d^2}\right) \ge 
 \sum_{u \in V} \left(\frac{2\deg_A(u)}{d} - \frac{\deg_A(u)^2}{d^2}\right) \\
    &= 2|A| - \sum_{u \in V} \left(\frac{\deg_A(u)^2}{d^2}\right).
    \end{align*}
    Let $\bf{A}$ be the adjacency matrix of $G$ and let $\Hat{\bf{A}} = \frac{1}{d}\bf{A}$ be its normalized version. Consider the vector $\Hat{\bf{A}} \ones_A$. It's $u^\text{th}$ coordinate will be equal to
    \[
    \left(\Hat{\bf{A}} \ones_A\right)_u = \frac{\deg_A(u)}{d}.
    \]
    Then
    \[
    \sum_{u \in V} \left(\frac{\deg_A(u)^2}{d^2}\right) = \left(\Hat{\bf{A}} \ones_A\right)^\top \left(\Hat{\bf{A}} \ones_A\right) = \ones_A^\top \bf{A}^2\ones_A.
    \]
    Next, we express $\ones_A$ as a sum of (1) it's projection onto $\ones$ and (2) component orthogonal to $\ones$. More precicely
    \[
    \ones_A = \ones_A^{\parallel} + \ones_A^{\perp},
    \]
    where $\ones_A^{\parallel} = \frac{\langle \ones_A, \ones \rangle}{\norm{\ones}_2^2} \ones = \frac{|A|}{n} \ones$ and $\langle \ones_A^\perp, \ones \rangle = 0$. Then
    \begin{align*}
    \sum_{u \in V} \left(\frac{\deg_A(u)^2}{d^2}\right) &= \ones_A^\top \bf{A}^2\ones_A 
    \intertext{since $\ones_A^{\parallel} $ and $\ones_A^{\perp}$ are orthogonal}
    &= (\ones_A^\parallel)^\top \Hat{\bf{A}}^2 (\ones_A^\parallel) + (\ones_A^\perp)^\top \Hat{\bf{A}}^2 (\ones_A^\perp) \\
    &= \frac{|A|^2}{n^2}\ones^\top \Hat{\bf{A}}^2 \ones + (\ones_A^\perp)^\top \Hat{\bf{A}}^2 (\ones_A^\perp) 
    \intertext{since $\ones$ is an eigenvector of $\Hat{\bf{A}}$}
    &= \frac{|A|^2}{n} + (\ones_A^\perp)^\top \Hat{\bf{A}}^2 (\ones_A^\perp)
    \intertext{since all eigenvalues of $\bf{A}$ are upper bounded by $\lambda$ in absolute value}
    &\le \frac{|A|^2}{n} + \lambda^2 (\ones_A^\perp)^\top  (\ones_A^\perp) \\
    &= \frac{|A|^2}{n} + \lambda^2 \norm{\ones_A^{\perp}}_2^2.
    \end{align*}
    By Pythagorean theorem,
    \[
    \norm{\ones_A}^2 = \norm{\ones_A^\parallel}^2 + \norm{\ones_A^\perp}^2 \Rightarrow \norm{\ones_A^\perp}^2 = \norm{\ones_A}^2 - \norm{\ones_A^\parallel}^2 = |A| - \frac{|A|^2}{n}.
    \]
    Hence,
    \[
    \sum_{u \in V} \left(\frac{\deg_A(u)^2}{d^2}\right) \le \frac{|A|^2}{n} + \lambda^2 \left(|A| - \frac{|A|^2}{n}\right) = \left(1 - \lambda^2\right)\frac{|A|^2}{n} + \lambda^2|A|.
    \]
    Therefore,
    \[
    \E[|A_{t+1}| \mid A_{t} = A ] \ge 2|A| - \left(1 - \lambda^2\right)\frac{|A|^2}{n} - \lambda^2|A| = |A|\left(1 + (1-\lambda^2)\left(1- \frac{|A|}{n}\right)\right).
    \]
\end{proof}

\pgfplotsset{compat=1.17}
\usepgfplotslibrary{fillbetween}
\newcommand\samples{1}

\begin{figure}
    \centering
    \begin{tikzpicture}[declare function={f(\x)=\x*(1 + (1-pow(0.3,2))*(1 - \x/100));}]
    \begin{axis}[width=12cm,
                 height=5cm,
                 axis x line=middle,
                 axis y line=left,
                 enlarge x limits=0.2,
                 enlarge y limits=0.02,
                 y axis line style ={opacity=0},
                 ytick=\empty,
                 xtick={0,20, 70,100},
                 xticklabels={,$\log(n)$, $9/10n$, },
                 x tick style={thick},
                 xtick align=outside,
                 tickwidth=0.25cm,
                 x tick label style={rotate=90,anchor=east},
                 xlabel={$|A_t|$},
                 ylabel={$\E\left[|A_{t+1}| \mid A_t\right]$}, 
                 very thick]
    
    \addplot[name path=graph, ultra thick, domain=0:100, smooth, samples=100*\samples]{f(x)}; 
    \addplot[name path=axis, very thick, domain=0:100, smooth, draw opacity =0, samples=100*\samples]{0}; 
    \addplot [thick, color=blue, fill=blue, fill opacity=0.2]
    fill between[of=graph and axis, soft clip={domain=0:20}];
    \addplot [thick, color=green, fill=green, fill opacity=0.2]
    fill between[of=graph and axis, soft clip={domain=20:70}];
    \addplot [thick, color=red, fill=red, fill opacity=0.2]
    fill between[of=graph and axis, soft clip={domain=70:100}];

    \path (0,0) coordinate (O) (20,0) coordinate (A) (70,0) coordinate (B) (100,0) coordinate (C);

    
    \end{axis}
    \draw [very thick, black, decorate,decoration={brace,amplitude=10pt,mirror},xshift=0.4pt,yshift=-0.4pt] (O) -- (A) node[black,anchor=east,midway,yshift=-0.3cm,rotate=90] {Phase I};
    \draw [very thick, black, decorate,decoration={brace,amplitude=10pt,mirror},xshift=0.4pt,yshift=-0.4pt] (A) -- (B) node[black,anchor=east,midway,yshift=-0.3cm,rotate=90] {Phase II};
    \draw [very thick, black, decorate,decoration={brace,amplitude=10pt,mirror},xshift=0.4pt,yshift=-0.4pt] (B) -- (C) node[black,anchor=east,midway,yshift=-0.3cm,rotate=90] {Phase III};

    \end{tikzpicture}    
\caption{Growth of BIPS, as given by the lower bound in Lemma~\ref{lemma:bips-growth}. In phase I, the active set is sublogarithmic. In phase II the active set is at least logarithmic and at most $9/10n$. In phase III, and the active set is at least $9/10n$ and reaches size $n$ at the end. Each phase takes $O(\log(n))$ rounds.}
    \label{fig:growth-bips}
\end{figure}


Due to space limitations, we only give the skeleton for the rest of the proof. We first observe that if $|A_t|/n \in 1 - \Omega_n(1)$, then $|A_{t+1}|$ is larger in expectation than $|A_t|$ by at least a constant factor. In other words, $|A_t|$ grows exponentially in expectation as long as $|A_t|/n$ is bounded away from $1$ by a constant. However, we need to prove a stronger statement than that, namely that $A_t$ grows \emph{with high probability}.

Notice that we can write $|A_{t+1}|$ as a sum of indicator random variables $\ind(u \in A_{t+1})$ as follows
\begin{equation}
\label{eq:indicators}
|A_{t+1}| = \sum_{u \in V} \ind(u \in A_{t+1}).
\end{equation}
Moreover, it follows from Definition~\ref{def:bips} that if we condition on $A_t$, the indicators $\ind(u \in A_{t+1})$ are \emph{independent} Bernoulli variables. Then, if $\E[|A_{t+1}| \mid A_t]$ is sufficiently large (but still far from $n$), we can show that $|A_{t+1}|$ is larger than $|A_t|$ by at least a constant fraction via \emph{Chernoff bounds}. This observation lays the base for the following proof structure (it's illustrated in Figure~\ref{fig:growth-bips}).

\paragraph{Phase I: Growth on small sets}

First, we show that the infected set $A_t$ grows to size $\Omega(\log(n))$ in at most $O(\log(n))$ time steps. The proof is similar in spirit to Chernoff bounds. We define $E_t$ to be the event that all $A_0, A_1, \ldots, A_t$ are less than some $K\log(n)$. By applying an exponential Markov inequality to $|A_t|$ and using indicator representation as in (\ref{eq:indicators}) we obtain $\Pr[E_t]$ is less than $\Pr[E_{t-1}]$ by at least a constant factor for every $t$. Then, by induction, probability $\Pr[E_t]$ is in $1/n^3$ (for a suitable choice of $t$). Formal statement is given below. 

\begin{restatable}{lemma}{PhaseOne}
    \label{lemma:phase1}
    Let $(A_t)_{t\ge 0}$ be $2$-BIPS process started from an arbitrary vertex $v\in V$. Then for any constant $K > 0$, there exist constant $c_1 = c_1(K)$ and $t_1 \le c_1\log(n)$ such that $|A_{t_1}| \ge K\log(n)$ with probability at least $1 - O\left(\frac{1}{n^3}\right)$. 
\end{restatable}

\paragraph{Phase II: Steady growth}

Now, suppose $|A_t| \ge K\log(n)$ for sufficiently large $K$. Then we can apply Chernoff bounds to (\ref{eq:indicators}) and obtain that with probability $1 - O(1/n^4)$, the active set grows by at least a constant factor. Then, in $O(\log(n))$ rounds, the active set grows to the size $\ge 9/10 n$ with probability $1 - O(1/n^3)$. The formal statement is given below.

\begin{restatable}{lemma}{PhaseTwo}
    \label{lemma:phase2}
    Let $(A_t)_{t\ge 0}$ be $2$-BIPS process. Suppose $t_1 > 0$ is such that $A_{t_1} \ge K\log(n)$ for large enough constant $K$. Then there exist a constant $c_2 = c_2(K)$ such that for $t_2 = t_1 + c_2\log(n)$, we have $|A_{t_2}| \ge 9/10 n$ with probability $1 - O\left(\frac{1}{n^3}\right)$.
\end{restatable}

\paragraph{Phase III: Infecting everyone}
Now, suppose $|A_t| \ge 9/10 n$. We look at $V\setminus A_t$ (i.e. set of uninfected nodes). We can show that with very high probability it shrinks by a constant factor in expectation at each time step. Then, in $O(\log(n))$ rounds, we will have $\E[|V\setminus A_t|] \le 1/n^3$, which by Markov inequality implies $\Pr[|V\setminus A_t| \ge 1] \le 1/n^3$.


\begin{restatable}{lemma}{PhaseThree}
    \label{lemma:phase3}
    Let $(A_t)_{t\ge 0}$ be $2$-BIPS process. Suppose $t_2 > 0$ is such that $A_{t_2} \ge \frac{9}{10}n$. Then there exist a constant $c_3$ such that for $t_3 = t_2 + c_3\log(n)$, we have $A_{t_3} = V$ with probability $1 - O\left(\frac{1}{n^3}\right)$.
\end{restatable}


\paragraph{Proof of Theorem~\ref{thm:main}} By gluing Lemmas~\ref{lemma:phase1},~\ref{lemma:phase2},~\ref{lemma:phase3} together with Lemma~\ref{lemma:dual} we obtain the proof of Theorem~\ref{thm:main}.


\begin{proof}[Proof of Theorem~\ref{thm:main}]
    Lemmas~\ref{lemma:phase1},~\ref{lemma:phase2},~\ref{lemma:phase3} together imply that there exist $c > 0$ such that for any $v\in V$ we have $\infec(v) \le t =  c\log(n)$ with probability $1 - O\left(\frac{1}{n^3}\right)$. 

    Then, by Lemma~\ref{lemma:dual} for $C = \{u\}$, we have 
    \[
    \Pr\left[\hit(u,v) > t \right] = \Pr\left[u \not \in A_t \mid A_0 = \{v\} \right] \le \Pr\left[\infec(v) > t\right] \in O\left(\frac{1}{n^3}\right).
    \]
    Then
    \[
    \Pr[\cov(u) > t] \le \sum_{v\in V} \Pr\left[\hit(u,v) > t \right] \in O\left(\frac{1}{n^2}\right).
    \]
    Then $\cov(u) < t$ with high probability. 

    Now we establish a bound in expectation. Consider the cobra walk at time $t$. Suppose it didn't cover $G$ yet, which happens with probability at most $O\left(\frac{1}{n^2}\right)$. Then we will follows a single branch of such a COBRA walk starting step $t$. It follows in distribution a simple random walk, which covers the graph in $O(n\log(n))$ expected steps. Then
    \[
    \E\left[\cov(u)\right] \in t + O\left(\frac{1}{n^2}\right) \cdot O(t + O(n\log(n))) \subseteq O(\log(n)).
    \]
\end{proof}

\printbibliography

% \appendix

% \section{Omitted proofs}

% \PhaseOne*
% \begin{sketch}
%     Let $m = K\log(n)$ and $\phi = \log\left(\frac{3 - \lambda}{2}\right) > 0$. For $t \ge 0$, let $E_{t}$ be the event that $|A_0|, |A_1|, \ldots, |A_t| < m+1$. We apply exponential Markov inequality
%     \begin{align*}    
%     \Pr[E_t] &= \Pr[E_{t-1}, |A_t| - |A_0| < m] =  \Pr[E_{t-1}, e^{-\phi(|A_t| - |A_0|)} > e^{-\phi m}] \\
%     &=\Pr[\ind(E_{t-1}) e^{-\phi(|A_t| - |A_0|)} > e^{-\phi m}] \le  e^{\phi m} \E[\ind(E_{t-1}) e^{-\phi(|A_t| - |A_0|)}]
%     \end{align*}
%     Let $G_t(\phi) =  \E[\ind(E_{t-1}) e^{-\phi(|A_t| - |A_0|)}]$. We aim to upper bound $G_t(\phi)$ in terms of $G_{t-1}(\phi)$. Let $\mathcal F_t = \sigma(A_0,\ldots, A_t)$ be the sigma algebra generated by $A_0,\ldots, A_t$. Using the law of total expectation and after grouping terms we get
%     \begin{equation}   
%     \label{eq:potential}
%     G_t(\phi) = \E\left[\underbrace{e^{-\phi(A_{t-1} - A_0)} \ind(E_{t-2})}_{\text{expression from $G_{t-1}(\phi)$}} \times \underbrace{\ind(A_{t-1} < m +1) \E[e^{-\phi(A_t - A_{t-1})} \mid \mathcal F_{t-1}]}_{\text{residual term}}\right]
%     \end{equation}
%     We will now show a universal upper bound on the residual term. Let $m$ be large enough so that $n = K\log(n) \le \frac{n}{2}$. We know that given $A_t$, the events $u \in A_{t+1}$ independent. Then, after conditioning on $A_{t-1}$, we may write $|A_t|$ as a sum of independent indicators $|A_t| = \sum_{u \in V} \ind(u \in A_{t})$. After some technical transitions we get
%     \[
%     \E[e^{-\phi(A_t - A_{t-1})} \mid \mathcal F_{t-1}] \le e^{\phi A_{t-1} - (e^{-\phi} - 1) \E[|A_{t}| \mid A_{t-1}]}
%     \]
%     Then, by Lemma~\ref{lemma:bips-growth} and by our choice of $\phi$ we can get $\E[|A_{t}| \mid A_{t-1}]] \ge e^{\phi} |A_{t-1}|$. Then
%     \[
%     \E[e^{-\phi(A_t - A_{t-1})} \mid \mathcal F_{t-1}] \le e^{|A_{t-1}|\left(\phi - e^{\phi} + 1\right)} \le e^{\phi - e^{\phi} + 1}
%     \]
%     Then, from~\ref{eq:potential}
%     \[
%     G_t(\phi) \le\E\left[e^{-\phi(A_{t-1} - A_0)} \ind(E_{t-2})\right] \times e^{\phi - e^{\phi} + 1} = G_{t-1}(\phi) \times e^{\phi - e^{\phi} + 1} 
%     \]
%     Hence, by induction 
%     \[
%     G_t(\phi) \le e^{t(\phi - e^{\phi} + 1)} 
%     \]
%     which implies
%     \[
%     \Pr[E_t] \le e^{\phi m} e^{t(\phi - e^{\phi} + 1)} 
%     \]
%     Since $\phi \in \Theta(1)$, we have $e^{\phi} -\phi - 1 \in \Theta(1)$. Then for every $m \in \Theta(\log(n))$ we can find $t \in \Theta(\log(n))$ such that $ \Pr[E_t] \le 1/n^3$. This implies the claim.
% \end{sketch}


% \PhaseTwo*
% \begin{sketch}
%     Let $\ind(u \in A_{t+1]})$ be the indicator random variable of the event $u \in A_{t+1}$. Then $A_{t+1} = \sum_{u \in V} \ind(u \in A_{t+1]})$. Note that conditioned on $A_t$, the indicators $\ind(u \in A_{t+1]})$ are independent, i.e., $|A_{t+1}|$ is a sum of independent Bernoulli random variables. Additionally, for any $|A_t| \le 9/10 n$ by Lemma~\ref{lemma:bips-growth}
%     \[
%     \E[|A_{t+1}| \mid A_t] \ge \left(1 + \frac{1 - \lambda^2}{10}\right) |A_{t}| \ge \left(1 + \frac{1 - \lambda^2}{10}\right) K\log(n)
%     \]
%     Then, taking $K$ large enough, we may conclude $\E[|A_{t+1}| \mid A_t] \ge (1 + \Omega(1))|A_{t}|$ with probability at least $1 - O\left(\frac{1}{n^4}\right)$ by Chernoff bounds. Then, by a union bound, in constant number of steps the size of $A_t$ doubles, and in logairthmic number of steps the size of $A_t$ will reach $9/10 n$ with probability at least 
%     \[1 - O\left(\frac{\log(n)}{n^4}\right) \subseteq 1 - O\left(\frac{1}{n^3}\right).\]
% \end{sketch}

% \PhaseThree*
% \begin{sketch}
%     Let $A_t '= V\setminus A_t$. By Lemma~\ref{lemma:bips-growth} 
%     \[
%     \E[|A_{t+1}'| \mid A_t] \le n - |A_t|\left(1 - (1-\lambda^2)\left(1 - \frac{|A_t|}{n}\right)\right) = |A_t'| \left(1 - (1-\lambda^2)|A_t|/n\right)
%     \]
%     Then
%     \begin{align*}
%     \E[|A_{t+1}'|] &= \E\left[|A_{t+1}'| \cdot \ind(|A_{t}| \ge 9/10n) \right] + \E\left[|A_{t+1}'|\cdot \ind(|A_{t}| < 9/10n)\right] \\
%                    &\le \E\left[|A_t'|\right] \left(1 - \frac{9(1-\lambda^2)}{10}\right) + n\Pr\left[|A_{t}| < 9/10n\right]
%     \end{align*}
%     Then
%     \[
%     \E[|A_{t_0+t}'|] \le \E[|A_{t_0}'|] \left(1 - \frac{9(1-\lambda^2)}{10}\right)^t + n\sum \Pr[|A_{t}| < 9/10n]
%     \]
%     Similarly to proof of Lemma~\ref{lemma:phase2}, we can prove by Chernoff bounds that if $|A_t| \ge 9/10n$, then $|A_{t +1} \ge 9/10|$ with probability $1 - O(n^8)$. Then for every $t \ge t_0$, we have $|A_t|\ge 9/10$ with probability at least $1 - O(n^7)$ by a union bound. 
%     Then,  by Markov inequality
%     \[
%     \Pr[|A_{t_0 + t}'| \ge 1] \le \E[|A_{t_0}'|] \left(1 - \frac{9(1-\lambda^2)}{10}\right)^t + n\sum \Pr[|A_{t}| < 9/10n]
%     \]
%     Picking $c(K)$ large enough, we can get for $t = c(K)\log(n)$
%     \[
%     \Pr[|A_{t_0 + t}'| \ge 1] \le \frac{1}{n^3}
%     \]
%     Which establishes the claim.
% \end{sketch}

\end{document}

%%% Local Variables:
%%% mode: latex
%%% TeX-master: t
%%% End: