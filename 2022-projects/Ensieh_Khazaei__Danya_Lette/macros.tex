%% Shamelessly adapted from a scribe template by Sanjeev Arora
%%%%%%%%%%%%%% Packages
% \usepackage[active,tightpage]{preview}
% \renewcommand{\PreviewBorder}{1in}
\usepackage{algorithmicx}
\usepackage{algorithm} %ctan.org\pkg\algorithms
\usepackage[noend]{algpseudocode}
\usepackage{amsfonts}
\usepackage{amsmath}
\usepackage{amssymb}
\usepackage{amstext}
\usepackage{amsthm}
\usepackage{bbm}
\usepackage{bm}
\usepackage{booktabs}
\usepackage[small]{caption}
\usepackage{color}
\usepackage{commath}
\usepackage{comment} 
\usepackage{enumerate}
\usepackage{enumitem}
\usepackage{epsfig} 
\usepackage{fancybox}
\usepackage{fullpage}
\usepackage{graphicx}
\usepackage{hyperref}
\usepackage{ifthen}
\usepackage{latexsym}
\usepackage{mdframed}
\usepackage{nicefrac}
\usepackage{pdfsync}
\usepackage{stmaryrd}
\usepackage{subfigure}
\usepackage{tabularx}
\usepackage{textcomp}
\usepackage{thm-restate}
\usepackage{thmtools}
\usepackage{url}
\usepackage{verbatim}
\usepackage{wrapfig}
\usepackage{xspace}
\usepackage{dsfont}
%%%%%%%%%%%%%% Use for definitions
\newcommand{\defeq}{\stackrel{\textup{def}}{=}}
\DeclareMathOperator{\vol}{\mathsf{vol}}
\DeclareMathOperator{\schur}{\mathsf{schur}}
%%%%%%%%%%%%%% Theorem Environments
\newtheorem{theorem}{Theorem}[section]
\newtheorem{problem}{Problem}
\newtheorem{lemma}[theorem]{Lemma}
\newtheorem{corollary}[theorem]{Corollary}
\newtheorem{conjecture}[theorem]{Conjecture}
\newtheorem{observation}[theorem]{Observation}
\newtheorem{proposition}[theorem]{Proposition}
\newtheorem{fact}[theorem]{Fact}
\newtheorem{claim}[theorem]{Claim}
\newtheorem{assumption}[theorem]{Assumption}
%\newtheorem{warning}[theorem]{Warning}
\theoremstyle{definition}
\newtheorem{definition}[theorem]{Definition}
\newtheorem{remark}[theorem]{Remark}
\newtheorem{example}[theorem]{Example}
\newcommand{\eps}{\varepsilon}
\newenvironment{fminipage}%
  {\begin{Sbox}\begin{minipage}}%
  {\end{minipage}\end{Sbox}\fbox{\TheSbox}}
\newenvironment{algbox}[0]{\vskip 0.2in
\noindent 
\begin{fminipage}{6.3in}
}{
\end{fminipage}
\vskip 0.2in
}
%%%%%%%%%%%%%% Probability stuff
\DeclareMathOperator*{\pr}{\bf Pr}
\DeclareMathOperator*{\av}{\mathbbm{E}}
\DeclareMathOperator*{\var}{\bf Var}
% \newcommand\pr{\mathop{\mbox{\bf Pr}}}
% \newcommand\av{\mathop{\mbox{\bf E}}}
% \newcommand\var{\mathop{\mbox{\bf Var}}}
%%%%%%%%%%%%%% Matrix stuff
\newcommand{\tr}[1]{\mathop{\mbox{Tr}}\left({#1}\right)}
\newcommand{\diag}[1]{{\bf Diag}\left({#1}\right)}
%% Notation for integers, natural numbers, reals, fractions, sets, cardinalities
%%and so on
\newcommand{\nfrac}[2]{\nicefrac{#1}{#2}}
\def\abs#1{\left| #1 \right|}
\renewcommand{\norm}[1]{\ensuremath{\left\lVert #1 \right\rVert}}
\newcommand{\floor}[1]{\left\lfloor\, {#1}\,\right\rfloor}
\newcommand{\ceil}[1]{\left\lceil\, {#1}\,\right\rceil}
\newcommand{\pair}[1]{\left\langle{#1}\right\rangle} %for inner product
\newcommand\B{\{0,1\}}      % boolean alphabet  use in math mode
\newcommand\bz{\mathbb Z}
\newcommand\nat{\mathbb N}
\newcommand\rea{\mathbb R}
\newcommand\R{\mathbb R}
\newcommand\com{\mathbb{C}}
\newcommand\plusminus{\{\pm 1\}}
\newcommand\Bs{\{0,1\}^*}   % B star use in math mode
\newcommand{\V}[1]{\mathbf{#1}} % Used to denote bold commands
                                % e.g. vectors, matrices
\DeclareRobustCommand{\fracp}[2]{{#1 \overwithdelims()#2}}
\DeclareRobustCommand{\fracb}[2]{{#1 \overwithdelims[]#2}}
\newcommand{\marginlabel}[1]%
{\mbox{}\marginpar{\it{\raggedleft\hspace{0pt}#1}}}
\newcommand\card[1]{\left| #1 \right|} %cardinality of set S; usage \card{S}
\renewcommand\set[1]{\left\{#1\right\}} %usage \set{1,2,3,,}
\renewcommand\complement{\ensuremath{\mathsf{c}}}
% \newcommand\poly{\mbox{poly}}  %usage \poly(n)
\DeclareMathOperator{\poly}{poly}
\newcommand{\comp}[1]{\overline{#1}}
\newcommand{\smallpair}[1]{\langle{#1}\rangle}
\newcommand{\ol}[1]{\ensuremath{\overline{#1}}\xspace}
%%%%%%%%%%%%%% Mathcal shortcuts
\newcommand\calA{\mathcal{A}}
\newcommand\calB{\mathcal{B}}
\newcommand\calC{\mathcal{C}}
\newcommand\calD{\mathcal{D}}
\newcommand\calE{\mathcal{E}}
\newcommand\calF{\mathcal{F}}
\newcommand\calG{\mathcal{G}}
\newcommand\calH{\mathcal{H}}
\newcommand\calI{\mathcal{I}}
\newcommand\calJ{\mathcal{J}}
\newcommand\calK{\mathcal{K}}
\newcommand\calL{\mathcal{L}}
\newcommand\calM{\mathcal{M}}
\newcommand\calN{\mathcal{N}}
\newcommand\calO{\mathcal{O}}
\newcommand\calP{\mathcal{P}}
\newcommand\calQ{\mathcal{Q}}
\newcommand\calR{\mathcal{R}}
\newcommand\calS{\mathcal{S}}
\newcommand\calT{\mathcal{T}}
\newcommand\calU{\mathcal{U}}
\newcommand\calV{\mathcal{V}}
\newcommand\calW{\mathcal{W}}
\newcommand\calX{\mathcal{X}}
\newcommand\calY{\mathcal{Y}}
\newcommand\calZ{\mathcal{Z}}
%%%%%%%%%%%%%% {{{ authornotes }}}
\definecolor{Mygray}{gray}{0.8}
 \ifcsname ifcommentflag\endcsname\else
  \expandafter\let\csname ifcommentflag\expandafter\endcsname
                  \csname iffalse\endcsname
\fi
\ifnum\showauthornotes=1
\newcommand{\todo}[1]{\colorbox{Mygray}{\color{red}#1}}
\else
\newcommand{\todo}[1]{}
\fi
\ifnum\showauthornotes=1
\newcommand{\Authornote}[2]{{\sf\small\color{red}{[#1: #2]}}}
\newcommand{\Authoredit}[2]{{\sf\small\color{red}{[#1]}\color{blue}{#2}}}
\newcommand{\Authorcomment}[2]{{\sf \small\color{gray}{[#1: #2]}}}
\newcommand{\Authorfnote}[2]{\footnote{\color{red}{#1: #2}}}
\newcommand{\Authorfixme}[1]{\Authornote{#1}{\textbf{??}}}
\newcommand{\Authormarginmark}[1]{\marginpar{\textcolor{red}{\fbox{%\Large
#1:!}}}}
\else
\newcommand{\Authornote}[2]{}
\newcommand{\Authoredit}[2]{}
\newcommand{\Authorcomment}[2]{}
\newcommand{\Authorfnote}[2]{}
\newcommand{\Authorfixme}[1]{}
\newcommand{\Authormarginmark}[1]{}
\fi
%%%%%%%%%%%%%% Logical operators
\newcommand\true{\mbox{\sc True}}
\newcommand\false{\mbox{\sc False}}
\def\scand{\mbox{\sc and}}
\def\scor{\mbox{\sc or}}
\def\scnot{\mbox{\sc not}}
\def\scyes{\mbox{\sc yes}}
\def\scno{\mbox{\sc no}}
%% Parantheses
\newcommand{\paren}[1]{\left({#1}\right)}
\newcommand{\sqparen}[1]{\left[{#1}\right]}
\newcommand{\curlyparen}[1]{\left\{{#1}\right\}}
\newcommand{\smallparen}[1]{({#1})}
\newcommand{\smallsqparen}[1]{[{#1}]}
\newcommand{\smallcurlyparen}[1]{\{{#1}\}}
%% short-hands for relational simbols
\newcommand{\from}{:}
\newcommand\xor{\oplus}
\newcommand\bigxor{\bigoplus}
\newcommand{\logred}{\leq_{\log}}
\def\iff{\Leftrightarrow}
\def\implies{\Rightarrow}
%% macros to write pseudo-code
\newlength{\pgmtab}  %  \pgmtab is the width of each tab in the
\setlength{\pgmtab}{1em}  %  program environment
 \newenvironment{program}{\renewcommand{\baselinestretch}{1}%
\begin{tabbing}\hspace{0em}\=\hspace{0em}\=%
\hspace{\pgmtab}\=\hspace{\pgmtab}\=\hspace{\pgmtab}\=\hspace{\pgmtab}\=%
\hspace{\pgmtab}\=\hspace{\pgmtab}\=\hspace{\pgmtab}\=\hspace{\pgmtab}\=%
\+\+\kill}{\end{tabbing}\renewcommand{\baselinestretch}{\intl}}
\newcommand {\BEGIN}{{\bf begin\ }}
\newcommand {\ELSE}{{\bf else\ }}
\newcommand {\IF}{{\bf if\ }}
\newcommand {\FOR}{{\bf for\ }}
\newcommand {\TO}{{\bf to\ }}
\newcommand {\DO}{{\bf do\ }}
\newcommand {\WHILE}{{\bf while\ }}
\newcommand {\ACCEPT}{{\bf accept}}
\newcommand {\REJECT}{\mbox{\bf reject}}
\newcommand {\THEN}{\mbox{\bf then\ }}
\newcommand {\END}{{\bf end}}
\newcommand {\RETURN}{\mbox{\bf return\ }}
\newcommand {\HALT}{\mbox{\bf halt}}
\newcommand {\REPEAT}{\mbox{\bf repeat\ }}
\newcommand {\UNTIL}{\mbox{\bf until\ }}
\newcommand {\TRUE}{\mbox{\bf true\ }}
\newcommand {\FALSE}{\mbox{\bf false\ }}
\newcommand {\FORALL}{\mbox{\bf for all\ }}
\newcommand {\DOWNTO}{\mbox{\bf down to\ }}
\let\originalleft\left
\let\originalright\right
\renewcommand{\left}{\mathopen{}\mathclose\bgroup\originalleft}
  \renewcommand{\right}{\aftergroup\egroup\originalright}
\def\pleq{\preccurlyeq}
\def\pgeq{\succcurlyeq}
\def\pge{\succ}
\def\ple{\prec}
\def\Approx#1{\approx_{#1}}
\def\Span#1{\textbf{Span}\left(#1  \right)}
\def\bvec#1{{\mbox{\boldmath $#1$}}}
\def\prob#1#2{\mbox{Pr}_{#1}\left[ #2 \right]}
\def\pvec#1#2{\vec{\mbox{P}}^{#1}\left[ #2 \right]}
\def\expec#1#2{{\mathbb{E}}_{#1}\left[ #2 \right]}
\def\var#1{\mbox{\bf Var}\left[ #1 \right]}
\newcommand{\E}{\mbox{{\bf E}}}
\def\defeq{\stackrel{\mathrm{def}}{=}}
\def\setof#1{\left\{#1  \right\}}
\def\sizeof#1{\left|#1  \right|}
\def\diag#1{\textsc{Diag}\left( #1 \right)}
\def\floor#1{\left\lfloor #1 \right\rfloor}
\def\ceil#1{\left\lceil #1 \right\rceil}
\def\dim#1{\mathrm{dim} (#1)}
\def\sgn#1{\mathrm{sgn} (#1)}
\def\union{\cup}
\def\intersect{\cap}
\def\Union{\bigcup}
\def\Intersect{\bigcap}
\def\abs#1{\left|#1  \right|}
\def\norm#1{\left\| #1 \right\|}
\def\smallnorm#1{\| #1 \|}
\newcommand\Ppsi{\boldsymbol{\mathit{\Psi}}}
\newcommand\PPsi{\boldsymbol{\mathit{\Psi}}}
\newcommand\ppsi{\boldsymbol{\mathit{\psi}}}
\newcommand\pphi{\boldsymbol{\mathit{\phi}}}
\newcommand\Llambda{\boldsymbol{\mathit{\Lambda}}}
\newcommand\PPi{\boldsymbol{\Pi}}
\newcommand\ppi{\boldsymbol{\pi}}
\newcommand\cchi{\boldsymbol{\chi}}
\newcommand\aalpha{\boldsymbol{\alpha}}
\newcommand\bbeta{\boldsymbol{\beta}}
\newcommand\ggamma{\boldsymbol{\gamma}}
\newcommand\ddelta{\boldsymbol{\delta}}
\newcommand\ttau{\boldsymbol{\tau}}
\def\aa{\pmb{\mathit{a}}}
\newcommand\bb{\boldsymbol{\mathit{b}}}
\newcommand\cc{\boldsymbol{\mathit{c}}}
\newcommand\dd{\boldsymbol{\mathit{d}}}
\newcommand\ee{\boldsymbol{\mathit{e}}}
\newcommand\ff{\boldsymbol{\mathit{f}}}
\let\ggreater\gg
\renewcommand\gg{\boldsymbol{\mathit{g}}}
\newcommand\hh{\boldsymbol{\mathit{h}}}
\newcommand\ii{\boldsymbol{\mathit{i}}}
\newcommand\jj{\boldsymbol{\mathit{j}}}
\newcommand\kk{\boldsymbol{\mathit{k}}}
\renewcommand\ll{\boldsymbol{\mathit{l}}}
\newcommand\pp{\boldsymbol{\mathit{p}}}
\newcommand\qq{\boldsymbol{\mathit{q}}}
\newcommand\bs{\boldsymbol{\mathit{s}}}
\newcommand\nn{\boldsymbol{\mathit{n}}}
\newcommand\rr{\boldsymbol{\mathit{r}}}
\renewcommand\ss{\boldsymbol{\mathit{s}}}
\def\tt{\boldsymbol{\mathit{t}}}
\newcommand\uu{\boldsymbol{\mathit{u}}}
\newcommand\vv{\boldsymbol{\mathit{v}}}
\newcommand\ww{\boldsymbol{\mathit{w}}}
\newcommand\yy{\boldsymbol{\mathit{y}}}
\newcommand\zz{\boldsymbol{\mathit{z}}}
\newcommand\xx{\boldsymbol{\mathit{x}}}
\newcommand\xxbar{\overline{\boldsymbol{\mathit{x}}}}
\newcommand\ddbar{\overline{\boldsymbol{\mathit{d}}}}
\renewcommand\AA{\boldsymbol{\mathit{A}}}
\newcommand\BB{\boldsymbol{\mathit{B}}}
\newcommand\CC{\boldsymbol{\mathit{C}}}
\newcommand\DD{\boldsymbol{\mathit{D}}}
\newcommand\EE{\boldsymbol{\mathit{E}}}
\newcommand\GG{\boldsymbol{\mathit{G}}}
\newcommand\II{\boldsymbol{\mathit{I}}}
\newcommand\JJ{\boldsymbol{\mathit{J}}}
\newcommand\KK{\boldsymbol{\mathit{K}}}
\newcommand\NN{\boldsymbol{\mathit{N}}}
\newcommand\MM{\boldsymbol{\mathit{M}}}
\newcommand\LL{\boldsymbol{\mathit{L}}}
\newcommand\PP{\boldsymbol{\mathit{P}}}
\newcommand\QQ{\boldsymbol{\mathit{Q}}}
\newcommand\RR{\boldsymbol{\mathit{R}}}
\renewcommand\SS{\boldsymbol{\mathit{S}}}
\newcommand\UU{\boldsymbol{\mathit{U}}}
\newcommand\WW{\boldsymbol{\mathit{W}}}
\newcommand\VV{\boldsymbol{\mathit{V}}}
\newcommand\XX{\boldsymbol{\mathit{X}}}
\newcommand\YY{\boldsymbol{\mathit{Y}}}
\newcommand\ZZ{\boldsymbol{\mathit{Z}}}
\newcommand\ffhat{\boldsymbol{\widehat{\mathit{f}}}}
\newcommand\gghat{\boldsymbol{\widehat{\mathit{g}}}}
\newcommand\rrhat{\boldsymbol{\widehat{\mathit{r}}}}
%\newcommand\rrhat{\hat{\boldsymbol{\mathit{r}}}}
\newcommand\xxhat{\boldsymbol{\widehat{\mathit{x}}}}
\newcommand\sshat{\boldsymbol{\widehat{\mathit{u}}}}
\newcommand\fftil{\boldsymbol{\widetilde{\mathit{f}}}}
\newcommand\ggtil{\boldsymbol{\widetilde{\mathit{g}}}}
\newcommand\ppsitil{\boldsymbol{\widetilde{\mathit{\psi}}}}
\newcommand\ddeltatil{\boldsymbol{\widetilde{\mathit{\delta}}}}
\newcommand\alphahat{{\widehat{{\alpha}}}}
\newcommand\ehat{{\widehat{{e}}}}
\newcommand\nhat{{\widehat{{n}}}}\newcommand\mhat{{\widehat{{m}}}}
\newcommand\that{{\widehat{{t}}}}
\newcommand\uhat{{\widehat{{u}}}}
\newcommand\vhat{{\widehat{{v}}}}
\newcommand\what{{\widehat{{w}}}}
\newcommand\Dhat{{\widehat{{\Delta}}}}
\newcommand\Dtil{{\widetilde{{\Delta}}}}
\newcommand\Dbar{{\bar{{\Delta}}}}
\newcommand\Gbar{{\overline{{G}}}}
\newcommand\Ghat{{\widehat{{G}}}}
\newcommand\Hbar{{\overline{{H}}}}
\newcommand\Hhat{{\widehat{{H}}}}
\newcommand\Htil{{\widetilde{{H}}}}
\newcommand\Vhat{{\widehat{{V}}}}
\newcommand\Ecal{{\mathcal{{E}}}}
\newcommand\Pcal{{\mathcal{{P}}}}
\newcommand{\sym}[1]{\mathrm{sym} (#1)}
\newcommand\Otil{\widetilde{O}}
\newcommand{\opt}{\textsc{Opt}}
\newcommand{\sign}[1]{\text{sign}\paren{#1}}
\newenvironment{tight_enumerate}{
\begin{enumerate}
 \setlength{\itemsep}{2pt}
 \setlength{\parskip}{1pt}
}{\end{enumerate}}
\newenvironment{tight_itemize}{
\begin{itemize}
 \setlength{\itemsep}{2pt}
 \setlength{\parskip}{1pt}
}{\end{itemize}}
\newcommand{\vone}{\boldsymbol{\mathbf{1}}}
\newcommand{\vzero}{\boldsymbol{\mathbf{0}}}
\newcommand\DDelta{\boldsymbol{\mathit{\Delta}}}
\newcommand{\etal}{\emph{et al.}}
\newcommand{\fftilde}{\boldsymbol{\widetilde{f}}}
% Theorem-type environments
% \theoremstyle{break} 
% \theoremheaderfont{\scshape}
% \theorembodyfont{\slshape}
% \newtheorem{Thm}{Theorem}[section]
% \newtheorem{Lem}[Thm]{Lemma}
% \newtheorem{Cor}[Thm]{Corollary}
% \newtheorem{Prop}[Thm]{Proposition}
% % \theoremstyle{plain} 
% % \theorembodyfont{\rmfamily} 
% \newtheorem{Ex}[Thm]{Exercise}
% \newtheorem{Exa}[Thm]{Example}
% \newtheorem{Rem}[Thm]{Remark}
% % \theorembodyfont{\itshape}
% \newtheorem{Def}[Thm]{Definition}
% \newtheorem{Conj}[Thm]{Conjecture}
% \newtheorem{Obs}[Thm]{Observation}
% \newtheorem{Ques}[Thm]{Question}
%\newenvironment{proof}{\noindent {\sc Proof:}}{$\Box$ \medskip} 
% \newenvironment{problems} % Definition of problems
%  {\renewcommand{\labelenumi}{\S\theenumi}
%  \begin{enumerate}}{\end{enumerate}}
%%%%%%%%%%%%%%%%% Proof Environments
\def\FullBox{\hbox{\vrule width 6pt height 6pt depth 0pt}}
%
%\def\qed{\ifmmode\qquad\FullBox\else{\unskip\nobreak\hfil
%\penalty50\hskip1em\null\nobreak\hfil\FullBox
%\parfillskip=0pt\finalhyphendemerits=0\endgraf}\fi}
\def\qedsketch{\ifmmode\Box\else{\unskip\nobreak\hfil
\penalty50\hskip1em\null\nobreak\hfil$\Box$
\parfillskip=0pt\finalhyphendemerits=0\endgraf}\fi}
%\newenvironment{proof}{\begin{trivlist} \item {\bf Proof:~~}}
 %  {\qed\end{trivlist}}
\newenvironment{proofsketch}{\begin{trivlist} \item {\bf
Proof Sketch:~~}}
  {\qedsketch\end{trivlist}}
\newenvironment{proofof}[1]{\begin{trivlist} \item {\bf Proof
#1:~~}}
  {\qed\end{trivlist}}
\newenvironment{claimproof}{\begin{quotation} \noindent
{\bf Proof of claim:~~}}{\qedsketch\end{quotation}}
%%%%%%%%%%%%%%%%%%%%%%%%%%%%%%%%%%%%%%%%%%%%%%%%%%%%%%%%%%%%%%%%%%%%%%%%%%%
%%%%%%%%%%%%%%%%%%%%%%%%%%%%%%%%%%%%%%%%%%%%%%%%%%%%%%%%%%%%%%%%%%%%%%%%%%%
\newlength{\tpush}
\setlength{\tpush}{2\headheight}
\addtolength{\tpush}{\headsep}
\newcommand{\handout}[5]{
   \noindent
   \begin{center}
   \framebox{ \vbox{ \hbox to \textwidth { {\bf \coursenum\ :\  \coursename} \hfill
#5 }
       \vspace{3mm}
       \hbox to \textwidth { {\Large \hfill #2  \hfill} }
       \vspace{1mm}
       \hbox to \textwidth { {\it #3 \hfill #4} }
     }
   }
   \end{center}
   \vspace*{4mm}
   \newcommand{\lecturenum}{#1}
   \addcontentsline{toc}{chapter}{Lecture #1 -- #2}
}
\newcommand{\lecturetitle}[4]{\handout{#1}{#2}{
  #3}{}{#4}}
\newcommand{\guestlecturetitle}[5]{\handout{#1}{#2}{Lecturer:
    #4}{Scribe: #3}{Lecture #1 - #5}}
%%%%%%%%%%%%%%%%%%%%%%%%%%%%%%%%%%%%%%%%%%%%%%%%%%%%%%%%%
%%% Commands to include figures
%% PSfigure
\newcommand{\PSfigure}[3]{\begin{figure}[t] 
  \centerline{\vbox to #2 {\vfil \psfig{figure=#1.eps,height=#2} }} 
  \caption{#3}\label{#1} 
  \end{figure}} 
\newcommand{\twoPSfigures}[5]{\begin{figure*}[t]
  \centerline{%
    \hfil
    \begin{tabular}{c}
        \vbox to #3 {\vfil\psfig{figure=#1.eps,height=#3}} \\ (a)
    \end{tabular}
    \hfil\hfil\hfil
    \begin{tabular}{c}
        \vbox to #3 {\vfil\psfig{figure=#2.eps,height=#3}} \\ (b)
    \end{tabular}
    \hfil}
  \caption{#4}
  \label{#5}
%  \sublabel{#1}{(a)}
%  \sublabel{#2}{(b)}
  \end{figure*}}
\newcounter{fignum}
% fig
%command to insert figure. usage \fig{name}{h}{caption}
%where name.eps is the postscript file and h is the height in inches
%The figure is can be referred to using \ref{name}
\newcommand{\fig}[3]{%
\begin{minipage}{\textwidth}
\centering\epsfig{file=#1.eps,height=#2}
\caption{#3} \label{#1}
\end{minipage}
}%
% ffigure
% Usage: \ffigure{name of file}{height}{caption}{label}
\newcommand{\ffigure}[4]{\begin{figure} 
  \centerline{\vbox to #2 {\hfil \psfig{figure=#1.eps,height=#2} }} 
  \caption{#3}\label{#4} 
  \end{figure}} 
% ffigureh
% Usage: \ffigureh{name of file}{height}{caption}{label}
\newcommand{\ffigureh}[4]{\begin{figure}[!h] 
  \centerline{\vbox to #2 {\vfil \psfig{figure=#1.eps,height=#2} }} 
  \caption{#3}\label{#4} 
  \end{figure}} 
% {{{ draftbox }}}
\ifnum\showdraftbox=1
\newcommand{\draftbox}{\begin{center}
  \fbox{%
    \begin{minipage}{2in}%
      \begin{center}%
%        \begin{Large}%
          \large\textsc{Working Draft}\\%
%        \end{Large}\\
        Please do not distribute%
      \end{center}%
    \end{minipage}%
  }%
\end{center}
\vspace{0.2cm}}
\else
\newcommand{\draftbox}{}
\fi
%% Complexity classes
\newcommand\p{\mbox{\bf P}\xspace}
\newcommand\np{\mbox{\bf NP}\xspace}
\newcommand\cnp{\mbox{\bf coNP}\xspace}
\newcommand\sigmatwo{\mbox{\bf $\Sigma_2$}\xspace}
\newcommand\ppoly{\mbox{\bf $\p_{\bf /poly}$}\xspace}
\newcommand\sigmathree{\mbox{\bf $\Sigma_3$}\xspace}
\newcommand\pitwo{\mbox{\bf $\Pi_2$}\xspace}
\newcommand\rp{\mbox{\bf RP}\xspace}
\newcommand\zpp{\mbox{\bf ZPP}\xspace}
\newcommand\bpp{\mbox{\bf BPP}\xspace}
\newcommand\ph{\mbox{\bf PH}\xspace}
\newcommand\pspace{\mbox{\bf PSPACE}\xspace}
\newcommand\npspace{\mbox{\bf NPSPACE}\xspace}
\newcommand\dl{\mbox{\bf L}\xspace}
\newcommand\ma{\mbox{\bf MA}\xspace}
\newcommand\am{\mbox{\bf AM}\xspace}
\newcommand\nl{\mbox{\bf NL}\xspace}
\newcommand\conl{\mbox{\bf coNL}\xspace}
\newcommand\sharpp{\mbox{\#{\bf P}}\xspace}
\newcommand\parityp{\mbox{$\oplus$ {\bf P}}\xspace}
\newcommand\ip{\mbox{\bf IP}\xspace}
\newcommand\pcp{\mbox{\bf PCP}}
\newcommand\dtime{\mbox{\bf DTIME}}
\newcommand\ntime{\mbox{\bf NTIME}}
\newcommand\dspace{\mbox{\bf SPACE}\xspace}
\newcommand\nspace{\mbox{\bf NSPACE}\xspace}
\newcommand\cnspace{\mbox{\bf coNSPACE}\xspace}
\newcommand\exptime{\mbox{\bf EXP}\xspace}
\newcommand\nexptime{\mbox{\bf NEXP}\xspace}
\newcommand\genclass{\mbox{$\cal C$}\xspace}
\newcommand\cogenclass{\mbox{\bf co$\cal C$}\xspace}
\newcommand\size{\mbox{\bf SIZE}\xspace}
\newcommand\sig{\mathbf \Sigma}
\newcommand\pip{\mathbf \Pi}
%%Computational problems
\newcommand\sat{\mbox{SAT}\xspace}
\newcommand\tsat{\mbox{3SAT}\xspace}
\newcommand\tqbf{\mbox{TQBF}\xspace}

\makeatletter
\newcommand{\vect}[1]{%
  \vbox{\m@th \ialign {##\crcr
  \vectfill\crcr\noalign{\kern-\p@ \nointerlineskip}
  $\hfil\displaystyle{#1}\hfil$\crcr}}}
\def\vectfill{%
  $\m@th\smash-\mkern-7mu%
  \cleaders\hbox{$\mkern-2mu\smash-\mkern-2mu$}\hfill
  \mkern-7mu\raisebox{-3.8pt}[\p@][\p@]{$\mathord\mathchar"017E$}$}
  
