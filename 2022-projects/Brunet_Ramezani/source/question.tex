% \documentclass{article}
% \usepackage[utf8]{inputenc}
% %%%%%%%%% options for the file macros.tex

% \def\showauthornotes{1}
% \def\showkeys{0}
% \def\showdraftbox{0}
% \let\pref=\prettyref

% 
\usepackage{amsmath, ulem, amssymb, amsfonts, hyperref, amsthm, listings, tcolorbox, bbm, xifthen, soul, mathtools}
\usepackage[margin=1in]{geometry}
\usepackage[linesnumbered,lined,boxed,commentsnumbered, ruled]{algorithm2e}
\hypersetup{
    colorlinks=true,
    linkcolor=blue,
    filecolor=magenta,      
    citecolor=magenta,      
    urlcolor=cyan,
}
\usepackage[
    backend=biber,
    style=alphabetic,
    sorting=ynt,
    backref=true
]{biblatex}
\usepackage[shortlabels]{enumitem}
\usepackage{cleveref}

\newtheorem{theorem}{Theorem}[section]
\newtheorem{lemma}[theorem]{Lemma}
\newtheorem{corollary}[theorem]{Corollary}
\newtheorem{claim}[theorem]{Claim}
\newtheorem{fact}[theorem]{Fact}
\newtheorem{open-problem}[theorem]{Open Problem}

\theoremstyle{definition}

\newtheorem{definition}[theorem]{Definition}
\newtheorem{example}[theorem]{Example}

\newtheorem{remark}{Remark}
\newtheorem{question}{Question}

\newcommand{\V}[1]{\mathbf{#1}\ignorespaces}
\renewcommand\AA{\boldsymbol{\mathit{A}}}
\newcommand\LL{\boldsymbol{\mathit{L}}}
\newcommand\MM{\boldsymbol{\mathit{M}}}
\newcommand\II{\boldsymbol{\mathit{I}}}
\newcommand\JJ{\boldsymbol{\mathit{J}}}
\newcommand\KK{\boldsymbol{\mathit{K}}}

\newcommand{\ab}[1]{\langle #1 \rangle}
\renewcommand{\subset}{\subseteq}
\newcommand{\dsquare}{\mathbin{\raisebox{.2ex}{
    \hspace{-.4em}$\bigcirc$\hspace{-.75em}{\rm s}\hspace{.15em}}}}
\newcommand{\from}{\overset{R}{\leftarrow}}
\newcommand{\1}{\mathbbm{1}}

\DeclareMathOperator{\USTCON}{USTCON}
\DeclareMathOperator{\STCON}{STCON}

% %%%%%%%%% Authornotes
% \newcommand{\Snote}{\Authornote{S}}
% \newcommand{\Scomment}{\Authorcomment{S}}
% \newcommand{\Sfnote}{\Authorfnote{S}}


% \title{Bounding the norm of the residual of the approximate solution}
% \author{Marc-Etienne Brunet}
% \date{November 2022}

% \begin{document}
% \maketitle
I struggled to understand the proof of a claim in the section about approximate linear regression
in the reference that you recommended~\cite[page 31]{DBLP:journals/corr/Woodruff14}.
Here is a \href{https://arxiv.org/pdf/1411.4357.pdf}{link} to the pdf for convenience.
He says that if we let $S$ be the span of the columns of $A$ union $b$,
and we have a JL transform, $\Pi$, for which
\begin{align}
\label{eq:bounds}
    \forall y \in S ~~ (1-\epsilon)\norm{y} \leq \norm{\Pi y} \leq (1+\epsilon)\norm{y} 
\end{align}
then it follows that 
\begin{align}
    \norm{b - A\tilde{w}} \leq (1 + \epsilon) \norm{b - Aw^*},
\end{align}
where 
\begin{align}
    w^* = \argmin_w \norm{b - Aw} & & \tilde{w} = \argmin_w \norm{\Pi(b - Aw)}.
\end{align}
We do bound the error, however, from what I can tell, 
if we only invoke the bounds of Equation~\ref{eq:bounds},
and don't explicitly consider that $\Pi$ is a linear transform, 
we get that
\begin{align*}
    \norm{b - A\tilde{w}} 
        &\leq \norm{b - Aw^*} + \epsilon \norm{b - Aw^*} + \epsilon \norm{b - A\tilde{w}} \\
        &\leq (1 + \epsilon) \norm{b - Aw^*} + \epsilon \norm{b - A\tilde{w}} \\ 
        &\leq \frac{(1 + \epsilon)}{(1 - \epsilon)} \norm{b - Aw^*} \\
        &\leq (1 + 2\epsilon + O(\epsilon^2))\norm{b - Aw^*}
\end{align*}
See the figure on the next page for a visual.
\begin{figure}
    \centering
    \includegraphics[width=15cm]{figures/residual_bound.png}
    \caption{Plot of bounds on the residual.}
\end{figure}
I know that with the above bound we could rescale $\epsilon$ by a constant
and keep the same order of the embedding dimension.
My worse case analysis in the figure is also avoided due to the 
fact that $\Pi$ is a linear transform. 
There's no way the $\tilde{w}$ could be in that position without non-linearities in the embedding.
(And we know from the latter proof, we can get away with far fewer rows in the embedding.)
But I still don't see how the $(1+\epsilon)$ bound 
simply ``follows'' from Equation~\ref{eq:bounds}.
I can't tell if David's proof misses this subtlety, 
or if he's implicitly invoking something else that I don't see. 
Is that clear to you?

% \bibliographystyle{apalike}
% \bibliography{bibs}
% \end{document}