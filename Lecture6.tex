\documentclass[11pt]{article}
%%%%%%%%% options for the file macros.tex

\def\showauthornotes{1}
\def\showkeys{0}
\def\showdraftbox{0}
% \allowdisplaybreaks[1]


\usepackage{amsmath, ulem, amssymb, amsfonts, hyperref, amsthm, listings, tcolorbox, bbm, xifthen, soul, mathtools}
\usepackage[margin=1in]{geometry}
\usepackage[linesnumbered,lined,boxed,commentsnumbered, ruled]{algorithm2e}
\hypersetup{
    colorlinks=true,
    linkcolor=blue,
    filecolor=magenta,      
    citecolor=magenta,      
    urlcolor=cyan,
}
\usepackage[
    backend=biber,
    style=alphabetic,
    sorting=ynt,
    backref=true
]{biblatex}
\usepackage[shortlabels]{enumitem}
\usepackage{cleveref}

\newtheorem{theorem}{Theorem}[section]
\newtheorem{lemma}[theorem]{Lemma}
\newtheorem{corollary}[theorem]{Corollary}
\newtheorem{claim}[theorem]{Claim}
\newtheorem{fact}[theorem]{Fact}
\newtheorem{open-problem}[theorem]{Open Problem}

\theoremstyle{definition}

\newtheorem{definition}[theorem]{Definition}
\newtheorem{example}[theorem]{Example}

\newtheorem{remark}{Remark}
\newtheorem{question}{Question}

\newcommand{\V}[1]{\mathbf{#1}\ignorespaces}
\renewcommand\AA{\boldsymbol{\mathit{A}}}
\newcommand\LL{\boldsymbol{\mathit{L}}}
\newcommand\MM{\boldsymbol{\mathit{M}}}
\newcommand\II{\boldsymbol{\mathit{I}}}
\newcommand\JJ{\boldsymbol{\mathit{J}}}
\newcommand\KK{\boldsymbol{\mathit{K}}}

\newcommand{\ab}[1]{\langle #1 \rangle}
\renewcommand{\subset}{\subseteq}
\newcommand{\dsquare}{\mathbin{\raisebox{.2ex}{
    \hspace{-.4em}$\bigcirc$\hspace{-.75em}{\rm s}\hspace{.15em}}}}
\newcommand{\from}{\overset{R}{\leftarrow}}
\newcommand{\1}{\mathbbm{1}}

\DeclareMathOperator{\USTCON}{USTCON}
\DeclareMathOperator{\STCON}{STCON}
\allowdisplaybreaks

\usepackage{tikz}

\usepackage[
    backend=biber,
% giveninits=true,
% natbib=true,
    style=alphabetic,
    url=false, 
 %   doi=true,
    hyperref,
    backref=true,
    backrefstyle=none,
    maxbibnames=10,
    sortcites
]{biblatex}
\addbibresource{papers.bib}

%%%%%%%%% Authornotes
\newcommand{\Snote}{\Authornote{S}}

\newenvironment{tight_enumerate}{
\begin{enumerate}
 \setlength{\itemsep}{2pt}
 \setlength{\parskip}{1pt}
}{\end{enumerate}}
\newenvironment{tight_itemize}{
\begin{itemize}
 \setlength{\itemsep}{2pt}
 \setlength{\parskip}{1pt}
}{\end{itemize}}



\addbibresource{refs.bib}

%%%%%%%%%%%%%%%%%%%%%%%%%

\begin{document}

\newcommand{\coursenum}{{CSC 2421H}}
\newcommand{\coursename}{{Graphs, Matrices, and Optimization}}
\newcommand{\courseprof}{Sushant Sachdeva}

\lecturetitle{6}{Electrical Networks and their Applications to Random Walks}{Calum MacRury}{October 22, 2018}

\section{Electrical Networks}

Let us suppose that $G=(V,E)$ is an undirected graph, for which $|V|=n$ and $|E|=m$. It will be convenient
to assume in this section that its edges are of the form $\{u,v\}$ for $u,v \in V$, highlighting
the fact that they are undirected. We shall later replace each edge $e=\{u,v\} \in E$,
with one of $(u,v)$ or $(v,u)$ to indicate the direction we wish to assign to $e$. In the
former case, we say that $u$ is directed to $v$, and in the latter case, we say that $v$ is directed
to $u$.

If we additionally specify a pair of distinct vertices $s,t \in V$, then we may consider an \textit{electrical
current} on $G$. Formally, this is a vector $\V{f} \in \rea^{E}$. Intuitively, the current assigned
to an edge should also have a direction associated to it. In order to specify these directions, 
we first assign an arbitrary orientation to the edges of $G$. If $\{u,v\} \in E$, then we can orient (direct) it from $u$ to $v$ by replacing it with the directed edge $(u,v)$,
and by writing $u \rightarrow v$. We then adopt the following convention for determining the direction
of the $|f_{e}|$ units of current for the (directed) edge $e=(u,v) \in \tilde{E}$:

\begin{itemize}
\item If $f_{u,v} >0$, then we say that $|f_{u,v}|$ units of current move from $u$ to $v$.
\item If $f_{u,v} < 0$, then we that $|f_{u,v}|$ units of current move from $v$ to $u$.
\item If $f_{u,v}=0$, then $0$ units of current move in either direction.

\end{itemize}

Using this convention, the vector $\V{f} \in \rea^{E}$ is sufficient to describe the directions of the electrical current 
on $G$. 

Now that we have formalized our notation for representing electrical currents, we consider a
number of \textit{conservation conditions} known as Kirchoff's Laws. Intuitively, these laws
say that for each \textit{internal node} (i.e $v \neq s,t$), the total amount of current entering
$v$ is equal to the total amount of current leaving it. That is, formally we have

\[
	\sum_{u: v \rightarrow u} f_{u,v} - \sum_{w: w \rightarrow v} f_{w,v}=0,
\]

for each $v \in V$, $v \neq s,t$.

Of course, if we consider $s$ to be the \textit{source} of the circuit, and
$t$ to be its \textit{sink}, then we can generalize these constraints to include
$s$ and $t$ as well. To do so, let us suppose that $\gamma \ge 0$ is the total
external current entering the source $s$. In this case, we have that

\[ \sum_{u: v \rightarrow u} f_{u,v} - \sum_{w: w \rightarrow v} f_{w,v}=
 \begin{cases}
 \gamma & \text{if $v = s$} 	\\
-\gamma & \text{if $v=t$}		\\
	0  & \text{otherwise}
\end{cases}
\]


For our purposes, we shall normalize and assume that the amount of current 
entering $s$ is $1$; that is, $\gamma=1$. In this case, there exists a 
unique matrix $\V{B} \in \rea^{n \times m}$ encoding the constraints we
impose on $\V{f}$. We have that $\V{f}$ satisfies the conservation laws, if and only
if 

\[
	\V{B}\V{f} = \mathbbm{1}_{s} - \mathbbm{1}_{t}.
\]

It is easy to check that for $(u,v) \in E$, if $\V{b}_{u,v}:= \mathbbm{1}_{u} - \mathbbm{1}_{v}$,
then 

\[
	\V{B} =[\V{b}_{e_1}\ldots \V{b}_{e_m}],
\]

provided $e_{1}, \ldots ,e_{m}$ are the (directed) edges of $G$.

Let us now additionally associate a \textit{resistance} $r_{e}$ to each edge
$e \in E$. If we consider the vector $\V{r}:=(r_{e})_{e \in E}$, then we 
can define the diagonal matrix $\V{R} \in \rea^{m \times m}$, by setting
$\diag{\V{R}}=\V{r}$.

We can use $\V{R}$ to define the \textit{energy} of the current $\V{f}$.
This is defined to be,

\[
	\sum_{e \in E} \frac{1}{2} r_{e} f_{e}^{2} = \frac{1}{2} \V{f}^{T} \V{R} \V{f}.
\]

With this definition, we can consider a natural optimization problem, in which
$\V{b}$ is taken to be a fixed vector within $\rea^{V}$, and $\V{B}$ is the matrix
defined above:

\begin{equation}\label{eqn:energy_optimization}
\begin{aligned}
&{\text{minimize}} & \frac{1}{2} \V{f}^{T}\V{R}\V{f}\\
& \text{subject to} &\V{B}\V{f}=\V{b} \\
&                             & \V{f} \in \rea^E
\end{aligned}
\end{equation}

We say that a current $\V{f}$ is an \textit{electrical current},
provided it is an optimum solution to OP \ref{eqn:energy_optimization}.

\begin{proposition}

If $\V{f}$ is an optimum solution to OP \ref{eqn:energy_optimization}
then $\V{R}\V{f}$ is orthogonal to $\Delta \in \rea^{m}$,
provided $\V{B}\Delta=0$. 


\end{proposition}


\begin{proof}

Given $\eps \ge 0$, we know that $\V{f} - \eps \Delta$ is a feasible
solution to OP \ref{eqn:energy_optimization}, as 

\[
	\V{B}(\V{f} - \eps \Delta) = \V{B}\V{f} - \eps \V{B}\Delta= b,
\]

since $\Delta$ was assumed to be in the kernel of $\V{B}$. On the other
hand, we know that $\V{f}$ is an optimum solution so,

\[
	\frac{1}{2}(\V{f} - \eps \Delta)^{T} \V{R} (\V{f} - \eps \Delta) \ge \frac{1}{2}\V{f}^{T} \V{R} \V{f}.
\]

After simplification, it follows that

\begin{equation}\label{eqn:epsilon_addition}
	- \eps( \Delta^{T} \V{R} \V{f} - \eps \frac{1}{2} \Delta^{T} \V{R} \Delta) \ge 0,
\end{equation}
for all $\eps \ge 0$. Dividing by $-\eps,$ we get
\begin{equation}
	\Delta^{T} \V{R} \V{f} - \eps \frac{1}{2} \Delta^{T} \V{R} \Delta \le 0,
\end{equation}


We know however that the entries of $\V{R}$ are nonnegative,
so as a consequence, $\Delta^{T} \V{R} \Delta \ge 0$. This implies that $\Delta^{T} \V{R} \V{f} \le 0$,
for otherwise the left-hand side of Equation \ref{eqn:epsilon_addition} is strictly positive for any $\eps > 0$.

Let us now suppose that in fact $\Delta^{T} \V{R} \V{f} <0$. In this case, it is clear that 
$\Delta^{T} \V{R} \Delta >0$, for otherwise Equation \ref{eqn:epsilon_addition} will not hold.
Observe that if we set $\eps_0 := -2 \frac{\Delta^{T} \V{R} \V{f}}{ \Delta^{T} \V{R} \Delta}$, then
$\eps_0 >0$, and for each $\eps > \eps_0$, we have that

\[
 	\eps(\Delta^{T} \V{R} \V{f} + \eps \frac{1}{2} \Delta^{T} \V{R} \Delta) >0,
\]

which contradicts Equation \ref{eqn:epsilon_addition}.
We may therefore conclude that $\Delta^{T} \V{R} \V{f} =0$, thus proving the orthogonality
of $\Delta$ and $\V{Rf}$.


\end{proof}


Suppose that we now consider the subspace $C:= \{ \Delta \in \rea^E: \V{B}\Delta =0\}$. Clearly,
$C$ is the kernel of the matrix $\V{B}$ (which is typically denoted by $\text{ker}(\V{B}))$. Moreover, as a corollary to the above proposition,
we get the following result:

\begin{corollary}

If $\V{f}$ is an optimal solution to Equation \ref{eqn:epsilon_addition}, then $\V{Rf}$
is orthogonal to the kernel of $\V{B}$; that is, $\V{f} \in C^{\perp}$ in the above
notation.

\end{corollary}

We shall now consider some properties of the subspace $C$. Before doing so,
let us observe a proposition regarding the matrix $\V{B}$.

\begin{proposition}

We have that $\text{rank}(\V{B})= n-1$ if and only if $G$ is connected (where
we consider $G$ as an undirected graph).

\end{proposition}

\begin{proof}

It is a standard result from linear algebra that both the row and column spaces of $\V{B}$
have the same dimension. This integer is defined precisely as the value of $\text{rank}(\V{B})$.

As a consequence of this result, we know that since $\V{B}^{T}$ is the transpose of $\V{B}$,
each matrix must have the same rank. It is therefore sufficient to show that the result holds
for the matrix $\V{B}^{T}$.

Observe that for any vector $\V{x} \in \rea^V$, we have that 

\[
	(\V{B}^{T})_{e} = x_{u} -x_{v},
\]

for each directed edge $e=(u,v)$ of $G$. In particular, this shows that
$\V{B}^{T} \mathbbm{1}_{V} =0$, and so the kernel of $\V{B}^{T}$
has dimension greater or equal to $1$. 

One can additionally show that if $G$ is connected,
then provided $\V{B}^{T}\V{x}=0$, we must have that $\V{x} \in \text{span}\{ \mathbbm{1}_{V} \}$.
In other words, the dimension of the kernel of $\V{B}^{T}$ is exactly $1$. In this case, 
the rank theorem for matrices implies that $\text{rank}(\V{B}^{T})=n-1$.

Generalizing the above results, we can prove that if $k(G) \ge 1$ is the number of components
of $G$, then we have that $\text{rank}(\V{B}^{T})=n-k(G)$ (here the components are formed
considering $G$ as an undirected graph). This completes both directions
of the proposition.

\end{proof}

\begin{lemma}

If $G$ is connected, then $\text{dim}(C)= m -(n-1)$.

\end{lemma}

\begin{proof}

This lemma is an immediate corollary of the previous result, combined with the rank theorem
for matrices.

\end{proof}

\begin{lemma}

If $Im(\V{B}^{T}):=\{ \V{B}^{T}\V{x}: \V{x} \in \rea^{V}\}$, namely the image of the matrix $\V{B}^{T}$,
then $Im(\V{B}^{T})=C^{\perp}=ker(B)^{\perp}$.


\end{lemma}

Before we prove this lemma, we remark that this statement is a property of all matrices - it is not
specific to $\V{B}$.

\begin{proof}


Observe that $\text{Im}(\V{B}^{T})$ is spanned by the column vectors
of $\V{B}^{T}$. As a result, a vector $\V{x} \in \rea^{E}$ is in $\text{Im}(\V{B})^{\perp}$
if and only if it is orthogonal to all the column vectors of $\V{B}^{T}$. On the other hand,
we know that $\V{x}$ satisfies this property if and only if
it is in the kernel of $\V{B}$. We may therefore conclude that,

\[
 	\text{Im}(\V{B}^{T})^{\perp} = C.
\]

Taking the orthogonal complement of both sides of the above equation, we may conclude that

\[
	\text{Im}(\V{B}^{T}) = C^{\perp},
\]

thus completing the proof.


\end{proof}

As a result of the preceding lemmas, we may conclude the following theorem:

\begin{theorem}\label{thm:energy_characterization}

If $\V{f}$ is an optimum solution to OP \ref{eqn:energy_optimization},
then there exists some $\V{x} \in \rea^{E}$ such that $\V{f} = \V{R}^{-1}\V{B}^{T}\V{x}$.


\end{theorem}

\begin{proof}

Observe that since $\V{f}$ is an optimum solution, we know that $\V{Rf}$ is orthogonal
to $C$. On the other hand, we know that $\text{Im}(B^{T})= C^{T}$, so there exists
some $\V{x} \in \rea^{E}$ for which $\V{Rf}= \V{R}^{-1}\V{B}^{T}\V{x}$.
As the matrix $\V{R}$ is invertible, the result thus holds.


\end{proof}

We know that if $\V{f}$ is an optimum solution to OP \ref{eqn:energy_optimization},
then we have that

\[
	\V{b} = \V{Bf}=\V{B}\V{R}^{-1}\V{B}^{T}\V{x},
\]

as $\V{Bf}=\V{b}$ and $\V{f}=\V{R}^{-1}\V{B}^{T}\V{x}$.
We can interpret the vector $\V{x}$ as specifying a \textit{voltage} of $x_{v}$ units
for each vertex $v \in V$. By applying $\V{R}^{-1}\V{B}^{T}$ to this vector $\V{x}$,
we can then recover the electrical current $\V{f}$ on the edges $G$, given that the
resistances are defined by $\V{R}$.

In the next section, we shall characterize exactly what these voltage vectors look like.
Before we discuss how this can be done, we first observe that the matrix $\V{B}\V{R}^{-1}\V{B}^{T}$
correponds to the Laplacian of a specific weighted graph we now describe.

\begin{lemma}

The matrix $\V{B}\V{R}^{-1}\V{B}^{T}$ is the Laplacian of $G=(V,E,w)$,
where $w_{e}:=\frac{1}{r_{e}}$ for each $e\in E$.


\end{lemma}


\begin{proof}

Observe that if $R=I$, the identity matrix, then

\begin{align*}
	\V{B}\V{B}^{T} &= \sum_{e =(u,v) \in E}( \mathbbm{1}_{u}-\mathbbm{1}_{v})( \mathbbm{1}_{u} - \mathbbm{1}_{v})^{T}	\\
			     &= \sum_{e = (u,v) \in E} L_{u,v},
\end{align*}

where $L_{u,v}$ is the Laplacian of the edge $e=(u,v) \in E$.

If the matrix $R$ is not the identity, then a similar computation shows that

\begin{align*}
	\V{B} \V{R}^{-1}\V{B}^{T} &= \sum_{e =(u,v) \in E}\frac{1}{r_{e}}( \mathbbm{1}_{u}-\mathbbm{1}_{v})( \mathbbm{1}_{u} - \mathbbm{1}_{v})^{T}	\\
			     &= \sum_{e = (u,v) \in E} \frac{1}{r_e} L_{u,v},
\end{align*}

where the matrix $\frac{1}{r_e} L_{u,v}$ is the Laplacian of the edge $e=(u,v)$ with weight $\frac{1}{r_{e}}$.



\end{proof}


\subsection{Moore-Pensoose Pseudo-Inverse}

Suppose that we are given a symmetric $n \times n$ matrix $\V{A}$ together with an $n$-vector $\V{b}$,
and we wish to solve the equation $\V{Ax} = \V{b}$ for solve variable $\V{x} \in \rea^{n}$. In the case in which $\V{A}$ is invertible, the vector $\V{x} = \V{A}^{-1}\V{b}$ is the unique solution to this equation.
When the matrix $\V{A}$ is not invertible, then we can define the \textit{Moore-Pensoose Pseudo-Inverse}
of $\V{A}$. If $\lambda_{1}, \ldots ,\lambda_{n}$ are the eigenvalues of $\V{A}$, then assume that
$\psi_{1},\ldots ,\psi_{n}$ are orthonormal eigenvectors of $\V{A}$. Using the spectral decomposition of $\V{A}$, we know that

\[
	\V{A} = \sum_{i=1}^{n} \lambda_{i} \psi_{i} \psi_{i}^{T}.
\]

We then define

\[
	\V{A}^{+}:= \sum_{i=1: \lambda_{i} \neq 0}^{n} \frac{1}{\lambda_{i}} \psi_{i} \psi_{i}^{T},
\]

as the \textit{pseudo-inverse} of $\V{A}$.


If we consider the specific case when we are given an undirected graph $G=(V,E)$,
then we can consider the \textit{pseudo-inverse} $\V{L}^{+}$ of its Laplacian
$\V{L}$. In this case, assuming that $\lambda_{1} \le \ldots \le \lambda_{n}$
are the eigenvalues of $\V{L}$, then we have that

\begin{equation}\label{eqn:pseudo_inverse_solver}
	\V{L}(\V{L}^{+}\V{b})= \sum_{i=1: \lambda_{i} >0}^{n} \psi_{i} \psi_{i}^{T} \V{b},
\end{equation}

where $\psi_{1},\ldots ,\psi_{n}$ are orthonormal eigenvectors of $\lambda_{1}, \ldots ,\lambda_{n}$.
Observe that if we define the matrix $\Pi:=\sum_{i=1: \lambda_{i} >0}^{n} \psi_{i} \psi_{i}^{T}$,
then $\Pi \in \rea^{V \times V}$. Moreover, $\Pi$ is an \textit{orthogonal projection} onto the subspace
spanned by $\beta:=\{\psi_{i}: \lambda_{i} >0 \; \text{and} \; 1\le i \le n\}$. That is, $\Pi^{2}=\Pi$,
$\Pi^{T}=\Pi$ and $\text{Im}(\Pi)= \text{span}(\beta)$.


For our purposes, we are particularly interested in the case when $G$ is connected.
If this is true, then we know that the kernel of $\V{L}$ is spanned by $\mathbbm{1}_{V}$, and so
$\lambda_{i} > 0$ for $i=2, \ldots ,n$. If we also assume that $\V{b}$ is orthogonal to $\mathbbm{1}_{V}$,
then $\V{b}\in \text{span}(\beta)$, and so $\Pi(\V{b})=\V{b}$.

Under these assumptions, observe that Equation \ref{eqn:pseudo_inverse_solver} simplifies to

\[
	\V{L}(\V{L}^{+}\V{b})= \V{b}.
\]

We may therefore conclude that $\V{x} = \V{L}^{+}\V{b}$ is a solution to the equation ``$\V{Lx}=\V{b}$''.

\begin{lemma}\label{lem:Laplacian_solution_characterization}

If $G=(V,E)$ is a connected graph, and $\V{b} \in \rea^{V}$ is orthogonal to $\mathbbm{1}_{V}$,
then 

\[
	\{ \V{x} \in \rea^{V}: \V{Lx}=\V{b}\} = \{\V{L}^{+}\V{b}+ \alpha \mathbbm{1}_{V}: \alpha \in \rea\}.
\]


\end{lemma}

\begin{remark}

If we orient the edges of $G$ and consider currents on the edges of $G$, then the condition on $\V{b}$
has a natural interpretation: We can think of the assumption $\V{b}^{T} \mathbbm{1}_{V}=0$ as enforcing
the constraint that the net current into the circuit must be $0$. For example, in the case of a single source-sink pair $(s,t)$,
the vector $\V{b}:=\mathbbm{1}_{s}-\mathbbm{1}_{t}$ has exactly one unit of current entering $s$ and one unit of current leaving
$t$. Clearly, the orthogonality condition is satisfied in this case.



\end{remark}


\begin{proof}

Observe that if $\alpha \in \rea$, then we have that

\[
	\V{L}(\V{L}^{+}\V{b}+\alpha \mathbbm{1}_{V})=\V{b},
\]

as $\mathbbm{1}_{V}$ is in the kernel of $\V{L}$. We may therefore
conclude that

\[
 \{\V{L}^{+}\V{b}+ \alpha \mathbbm{1}_{V}: \alpha \in \rea\} \subseteq \{ \V{x} \in \rea^{V}: \V{Lx}=\V{b}\}.
\]

To see the other inclusion, assume that $\V{x} \in \rea^{V}$ is such that
$\V{Lx}=\V{b}$. In this case, we have that $\V{Lx}=\V{b} = \V{L}(\V{L}^{+}\V{b})$.
Thus,

\[
	\V{L}( \V{L}^{+}\V{b}-\V{x}) = 0,
\]

and so $\V{L}^{+}\V{b}-\V{x} \in \text{ker}(\V{L})$. But $G$ was assumed to be
connected, so the kernel of $\V{L}$ is spanned by $\mathbbm{1}_{V}$. It follows that
there exists some $\alpha_{0} \in \rea$ such that $\V{L}^{+}\V{b}-\V{x} = \alpha_{0} \mathbbm{1}_{V}$.
Thus, $\V{x} = \V{L}^{+}\V{b}-\alpha_{0} \mathbbm{1}_{V}$, and so $\V{x} \in \{\V{L}^{+}\V{b}+ \alpha \mathbbm{1}_{V}: \alpha \in \rea\}$.
This implies the other direction of the inclusion, and so the statement holds.

\end{proof}

We conclude this section by remarking that if we orient the graph $G$
as in the previous section, and specify a resistance matrix $\V{R}$ on its edges,
then we can consider the special matrix defined by $\V{L} = \V{B}\V{R}^{-1}\V{B}^{T}$,
where the matrix $\V{B}$ is derived from the orientation on $G$. As we saw previously, $\V{L}$ is in fact a Laplacian matrix.
If we set $\V{b}:= \mathbbm{1}_{s} - \mathbbm{1}_{t}$ for some source-sink pair $(s,t) \in V$, then we can interpret
this vector as passing $1$ unit of current into $s$, and $1$ unit out of $t$. In particular, we know that $\mathbbm{1}^{T} \V{b}=0$.
Thus, if we consider solutions to the equation ``$\V{Lx}=\V{b}$'', we know that 

\[
	\{ \V{x} \in \rea^{V}: \V{Lx}=\V{b}\} = \{\V{L}^{+}\V{b}+ \alpha \mathbbm{1}_{V}: \alpha \in \rea\},
\]

by Lemma \ref{lem:Laplacian_solution_characterization}. Observe that if $\V{x} := \V{L}^{+}\V{b}+ \alpha \mathbbm{1}_{V}$
for some $\alpha \in \rea$, then $\V{x}$ specifies voltages on the vertices of $G$. Moreover, by applying $\V{R}^{-1}\V{B}^{T}$
to $\V{x}$, we can recover the current 

\[
	\V{f} = \V{R}^{-1}\V{B}^{T}(\V{L}^{+}\V{b}+ \alpha \mathbbm{1}_{V}).
\]

Of course, $\V{B}^{T}\mathbbm{1}_{V}=0$, so this means that $\V{f}=\V{R}^{-1}\V{B}^{T}(\V{L}^{+}\V{b})$. 
In particular, the current $\V{f}$ must be an optimum solution to OP \ref{eqn:energy_optimization} by Theorem
\ref{thm:energy_characterization}. In other words, any voltage solution of the above form induces an electrical current
through $G$; that is, a current whose energy is maximum.

\subsection{Random Walks}

Let us consider a weighted undirected graph $G=(V,E,\V{w})$, together with a pair of distinct
nodes $s,t \in V$. If we assume that $G$ is connected, and start a (non-lazy) random walk
at node $s$, then we can define $h(s,t)$ to be the expected number of steps
for the walk to reach $t$ for the first time. One can show that this value is finite,
no matter which nodes are chosen.

If we fix the node $t$, then we can define the vector $\V{h}_{t} \in \rea^{V}$,
where

\[
	\V{h}_{t}(s):=h(s,t),
\]

for each $s \in V(G)$. We first observe that for $x \neq t$,

\[
	\V{h}_{t}(x) = 1 + \sum_{y: \{x,y\} \in E} \frac{w_{x,y}}{\deg(x)} \V{h}_{t}(y),
\]

and $\V{h}_{t}(t)=0$. As a consequence, we know that for each $x \neq t$,

\[
	\deg(x) \V{h}_{t}(x) = \deg(x) + \sum_{y: \{x,y\} \in E} w_{x,y} \V{h}_{t}(y).
\]

In vector notation, provided $x\neq t$, we can express this as

\[
	(\V{D}\V{h}_{t})(x) = (\V{D}\mathbbm{1}_{V})(x) + (\V{A}\V{h}_{t})(x),
\]

provided $\V{A}$ is the adjacency matrix of $G$, and $\V{D}$ is its degree matrix. Observe then that

\[
	(\V{L}\V{h}_{t})(x)= (\V{D}\mathbbm{1}_{V})(x),
\]

for all $x\neq t$, where $\V{L}$ is the Laplacian of $G$. On the other hand, we know that

\[
	0 = \mathbbm{1}_{V}^{T} \V{L}\V{h}_{t}= \sum_{x \neq t}(\V{L}\V{h}_{t})(x) + (\V{L}\V{h}_{t})(t),
\]

as $\mathbbm{1}_{V}$ is in the kernel of $\V{L}$, and is thus orthogonal to $\V{L}\V{h}_{t}$ (check this). It follows that

\[
	(\V{L}\V{h}_{t})(t)= -\sum_{x\neq t}\deg(x) = \deg(t) - \mathbbm{1}^{T}\V{D}\mathbbm{1}.
\]

If we define $\V{b}:= \V{D}\mathbbm{1}_{V} - (\mathbbm{1}^{T}\V{D}\mathbbm{1})\mathbbm{1}_{t}$,
then this implies that

\[
	\V{L}\V{h}_{t}= \V{b}.
\]

We may therefore use Lemma \ref{lem:Laplacian_solution_characterization} to conclude that

\[
	\V{h}_{t} = \V{L}^{+}(\V{D}\mathbbm{1} - (\mathbbm{1}^{T}\V{D}\mathbbm{1})\mathbbm{1}_{t}) + \alpha\mathbbm{1},
\]

for some $\alpha \in \rea$, as $\V{b}$ is orthogonal to $\mathbbm{1}$. Observing that
$\V{h}_{t}(t)=0$, we may conclude that

\[
	\mathbbm{1}_{t}^{T}\V{h}_{t}= \mathbbm{1}_{t}^{T}( \V{L}^{+}\V{b}+\alpha \mathbbm{1})=0.
\]

Thus, $\alpha = -\mathbbm{1}_{t}^{T}\V{L}^{+}\V{b}$. If we fix some $s \in V$,
then this implies that
\begin{align*}
h(s,t) =& \mathbbm{1}_{s}^{T}\V{h}_{t} 		\\
	=& \mathbbm{1}_{s}^{T}\V{L}^{+}\V{b} + \alpha \mathbbm{1}_{s}^{T} \mathbbm{1}	\\
	=& \mathbbm{1}_{s}^{T}\V{L}^{+}\V{b} - \mathbbm{1}_{t}^{T} \V{L}^{+} \V{b}		\\
	=&(\mathbbm{1}_{s}-\mathbbm{1}_{t})^{T} \V{L}^{+}(\V{D}\mathbbm{1} - (\mathbbm{1}^{T}\V{D}\mathbbm{1})\mathbbm{1}_{t}),
\end{align*}

giving us a convenient expression for the hitting time vector $\V{h}_{t}$.

\subsection{Commute Time}


In addition to hitting times on undirected weighted graphs, we can also define commute
times. If $G=(V,E,\V{w})$ is connected, then provided $s,t \in V$, we can define
the commute time from $s$ to $t$ as

\[
	C(s,t):= h(s,t) +h(t,s).
\]

By our results from the previous section, we know that

\[
	h(s,t) = (\mathbbm{1}_{s}-\mathbbm{1}_{t})^{T} \V{L}^{+}(\V{D}\mathbbm{1} - (\mathbbm{1}^{T}\V{D}\mathbbm{1})\mathbbm{1}_{t})
\]

Thus, after simplification

\[
	C(s,t) = [(\mathbbm{1}_{s}-\mathbbm{1}_{t})^{T} \V{L}^{+} ( \mathbbm{1}_{s} - \mathbbm{1}_{t})](\mathbbm{1}^{T} \V{D}\mathbbm{1}).
\]

We remark that the vector  $\V{L}^{+} ( \mathbbm{1}_{s} - \mathbbm{1}_{t})$ can be interpreted as specifying voltages
on the vertices of $G$. If $1$ unit of current enters $s$ and $1$ unit leaves $t$, then an electrical current can be derived
from this vector (see the end of the previous section for details).






\printbibliography
\end{document}




%%% Local Variables:
%%% mode: latex
%%% TeX-master: t
%%% End:
