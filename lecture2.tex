\documentclass[11pt]{article}
%%%%%%%%% options for the file macros.tex

\def\showauthornotes{1}
\def\showkeys{0}
\def\showdraftbox{0}
% \allowdisplaybreaks[1]


\usepackage{amsmath, ulem, amssymb, amsfonts, hyperref, amsthm, listings, tcolorbox, bbm, xifthen, soul, mathtools}
\usepackage[margin=1in]{geometry}
\usepackage[linesnumbered,lined,boxed,commentsnumbered, ruled]{algorithm2e}
\hypersetup{
    colorlinks=true,
    linkcolor=blue,
    filecolor=magenta,      
    citecolor=magenta,      
    urlcolor=cyan,
}
\usepackage[
    backend=biber,
    style=alphabetic,
    sorting=ynt,
    backref=true
]{biblatex}
\usepackage[shortlabels]{enumitem}
\usepackage{cleveref}

\newtheorem{theorem}{Theorem}[section]
\newtheorem{lemma}[theorem]{Lemma}
\newtheorem{corollary}[theorem]{Corollary}
\newtheorem{claim}[theorem]{Claim}
\newtheorem{fact}[theorem]{Fact}
\newtheorem{open-problem}[theorem]{Open Problem}

\theoremstyle{definition}

\newtheorem{definition}[theorem]{Definition}
\newtheorem{example}[theorem]{Example}

\newtheorem{remark}{Remark}
\newtheorem{question}{Question}

\newcommand{\V}[1]{\mathbf{#1}\ignorespaces}
\renewcommand\AA{\boldsymbol{\mathit{A}}}
\newcommand\LL{\boldsymbol{\mathit{L}}}
\newcommand\MM{\boldsymbol{\mathit{M}}}
\newcommand\II{\boldsymbol{\mathit{I}}}
\newcommand\JJ{\boldsymbol{\mathit{J}}}
\newcommand\KK{\boldsymbol{\mathit{K}}}

\newcommand{\ab}[1]{\langle #1 \rangle}
\renewcommand{\subset}{\subseteq}
\newcommand{\dsquare}{\mathbin{\raisebox{.2ex}{
    \hspace{-.4em}$\bigcirc$\hspace{-.75em}{\rm s}\hspace{.15em}}}}
\newcommand{\from}{\overset{R}{\leftarrow}}
\newcommand{\1}{\mathbbm{1}}

\DeclareMathOperator{\USTCON}{USTCON}
\DeclareMathOperator{\STCON}{STCON}
\allowdisplaybreaks

\usepackage{tikz, mathtools}

\usepackage[
    backend=biber,
% giveninits=true,
% natbib=true,
    style=alphabetic,
    url=false, 
 %   doi=true,
    hyperref,
    backref=true,
    backrefstyle=none,
    maxbibnames=10,
    sortcites
]{biblatex}
\addbibresource{papers.bib}

%%%%%%%%% Authornotes
\newcommand{\Snote}{\Authornote{S}}

\newenvironment{tight_enumerate}{
\begin{enumerate}
 \setlength{\itemsep}{2pt}
 \setlength{\parskip}{1pt}
}{\end{enumerate}}
\newenvironment{tight_itemize}{
\begin{itemize}
 \setlength{\itemsep}{2pt}
 \setlength{\parskip}{1pt}
}{\end{itemize}}


%\newcommand\eqdef{\stackrel{\mathclap{\normalfont\mbox{def}}}{=}}
\newcommand\idenVec{\V{1}}
% \DeclareMathOperator{\vol}{Vol}
\newcommand\nuReq{y^\top\V{D}\V{1}=0}
\newcommand\nuFrac{\frac{y^\top\V{L}y}{y^\top\V{D}y}}


\addbibresource{refs.bib}

%%%%%%%%%%%%%%%%%%%%%%%%%

\begin{document}

\newcommand{\coursenum}{{CSC 2421H}}
\newcommand{\coursename}{{Graphs, Matrices, and Optimization}}
\newcommand{\courseprof}{Sushant Sachdeva}

\lecturetitle{2}{Clustering}{Liwen Lu}{17 Sep 2018}

\section{Clustering of a Graph and Eigenvalue}

Informally, ``clustering'' is grouping vertices such that there are more connectivity inside each group compare to between groups. 

\subsection{Conductance}
\begin{definition}
\emph{Volume}: let $S \subseteq V$, $\vol(S)\defeq\sum\limits_{v \in S} d_v$ where $d_v = \deg(v) = \V{D}_{v, v}$
\end{definition}

\begin{lemma}
Define the indicator of set $S$ as 
\begin{align*}
\idenVec_S (v)
&=\begin{cases}1,&v\in S\\0,&\text{otherwise}\end{cases}\\
\end{align*}
Then $$\idenVec_S^T\V{D}\idenVec_S = \vol(S)$$
\end{lemma}

\begin{definition}
Define \emph{Boundary of S} as $$E(S, \overline{S}) = \{(u,v) | (u,v) \in E, u\in S, v \in V\setminus S\},$$ where $\overline{S}$ is the complement set.
\end{definition}

\begin{definition}
Define \emph{conductance of $S$ measured in graph $G$} to be $$\phi_G(S) \defeq\frac{|E(S, \overline{S})|}{min\{\vol(S), \vol(V\setminus S)\}},$$
and define \emph{conductance of graph $G$} to be $$\phi(G) \defeq \min\limits_{\substack{S \subseteq V\\ S\neq \emptyset, V}}\phi_G(S).$$
\end{definition}
Computing $\phi(G)$ is called ``minimum conductance problem'', and it is famously NP hard. However, we can connect conductance and eigenvalues. 


\subsection{Normalized Laplacian Matrix}
Let $\nu_2=\min\limits_{\nuReq}\paren{\nuFrac}$. 

Define $x = \V{D}^{\frac{1}{2}}y$ (note $\V{D}^{\frac{1}{2}}$ is replacing each of the diagonal entries in diagonal matrix $\V{D}$ with square root), we can write $y^\top\V{D}y = x^\top x$.
\newpage
Note we are assuming the graph $G$ is connected, then diagonal of $D$ is strictly non-negative, so the mapping between $x$ and $y$ is a bijection, which gives us $x = \V{D}^{\frac{1}{2}}y \iff y = \V{D}^{-\frac{1}{2}}x \implies y^\top\V{D}y = x^\top x$. Then we can rewrite $\nu_2$ as $$\nu_2 = \min\limits_{x^\top\paren{\V{D}^{\frac{1}{2}}\idenVec}=0}\frac{x^\top\V{D}^{-\frac{1}{2}}\V{L}\V{D}^{-\frac{1}{2}}x}{x^\top x}.$$

\begin{definition}
Define \emph{Normalized Laplacian Matrix} as 
\begin{align*}
\V{N} &\defeq \V{D}^{-\frac{1}{2}}\V{L}\V{D}^{-\frac{1}{2}}
\intertext{We also claim that:}
\lambda_1(\V{N}) &= 0 \\
\psi_1 &= \V{D}^{\frac{1}{2}}\idenVec \\
\nu_2 
&= \min\limits_{x^\top
\psi_1=0}\frac{x^\top\V{N}x}{x^\top x} \\
&= \lambda_2(\V{N})
\end{align*}
\end{definition}

\subsection{Cheeger's Inequality}
\begin{theorem}
$$\frac{\nu_2}{2} \leq \phi(G) \leq \sqrt{2\nu_2}, \nu_2 \leq 2$$
\end{theorem}
For example, if $\nu_2 = 0.01 \implies 0.005 \leq \phi(G) \leq 0.14$ \\ 
We'll first prove the left hand side, $\frac{\nu_2}{2} \leq \phi(G)$.
\begin{proof}
By definition, we know $\exists S$ $s.t.$ $\phi(G) = \phi_G(S) = \frac{|E(S, \overline{S})|}{\vol(S)}$ and $\vol(S) \leq \vol(\overline{S})$. \\
Recall $\nu_2=\min\limits_{\nuReq}\paren{\nuFrac}$ and $\idenVec_S^\top\V{D}\idenVec_S = \vol(S)$, we have 
\begin{align*}
\idenVec^\top_S\V{L}\idenVec_S &= \sum\limits_{(u,v)\in E}\paren{\idenVec_S(u) - \idenVec_S(v)}^2\\
\intertext{note}
\paren{\idenVec_S(u) - \idenVec_S(v)}^2 
&= \begin{cases}1,&\idenVec_S(v)\neq\idenVec_S(v)\iff(u,v)\in E(S, \overline{S})\\0,&\text{otherwise}\end{cases}\\
\intertext{therefore}
\idenVec^\top_S\V{L}\idenVec_S 
&= \sum_{(u,v)\in E}\idenVec[(u,v)\in E(S, \overline{S})] = |E(S, \overline{S})|
\end{align*}

If we set $y = \idenVec_S$, it would minimize $\nuFrac$, however it may not satisfy $\nuReq$. In order to make $y$ satisfy $\nuReq$, let $y = \idenVec_S + c\idenVec$, note we still have $y^\top\V{L}y = |E(S, \overline{S})|$.
\begin{align*}
y^\top\V{D}\idenVec = 0 &\iff c = \frac{-\idenVec^\top_S\V{D}\idenVec}{\idenVec^\top\V{D}\idenVec} = -\frac{\vol(S)}{\vol(V)} \tag{by previous lemma} \\ 
y^\top\V{D}y 
&= \idenVec_S^\top\V{D}\idenVec_S + 2c\idenVec_S^\top \V{D}\idenVec + c^2\idenVec^\top\V{D}\idenVec \\
&= \idenVec_S^\top\V{D}\idenVec_S - \frac{\paren{\idenVec^\top_S\V{D}\idenVec}^2}{\idenVec^\top\V{D}\idenVec} \\ 
&= \vol(S) - \frac{\vol(S)^2}{\vol(V)}\\
&= \vol(S)\paren{1 - \frac{\vol(S)}{\vol(V)}} \\
&\geq \frac{\vol(S)}{2} \tag{since $\vol(S) \leq \vol(\overline{S}$)}\\ 
\intertext{Therefore}
\nuFrac
&= \frac{|E(S, \overline{S})|}{y^\top\V{D}y} \\ 
&\leq \frac{2|E(S, \overline{S})|}{\vol(S)} \\ 
&= 2\phi_G(S) \\ 
&= 2\phi(G) \tag{by assumption} \\ 
\implies \nu_2
&= \min\limits_{\nuReq}\nuFrac \leq 2\phi(G) \\
\implies \frac{\nu_2}{2} &\leq \phi(G)
\end{align*}
\end{proof}

\begin{lemma}
Given $y$ s.t. $\nuReq$, we can find a distribution on $t$ with $S_t \subseteq V$ s.t. $\vol(S_t) \leq\frac{\vol(V)}{2}$, and 
$$\frac{\mathbb{E}_t|E(S_t, \overline{S_t})|}{\mathbb{E}_t\vol(S_t)} \leq \sqrt{\frac{2y^\top\V{L}y}{y^\top\V{D}y}}$$
\end{lemma}

Let $t$ be independent choices, $P_t$ be distribution of $t$, $X_t$, $Y_t$ are variables depend on $t$ with $Y_t \geq 0$. \\ 
We can prove that $\exists t$ s.t. 
$$ \frac{X_t}{Y_t} \leq \frac{\mathbb{E}_tX_t}{\mathbb{E}_tY_t} = \frac{\sum_tP_tX_t}{\sum_tP_tY_t}$$
The proof is left as exercise. This implies that $\exists t$, s.t. $$\frac{|E(S_t, \overline{S_t})|}{\vol(S_t)} \leq \sqrt{\frac{2y^\top\V{L}y}{y^\top\V{D}y}}$$ \\ 
Furthermore, if $y$ was the minimizing vector for $\nu_2$, then $$\phi_S(S_t) = \frac{|E(S_t, \overline{S_t})|}{\vol(S_t)} \leq \sqrt{2\nu_2}$$

\printbibliography
\end{document}


%%% Local Variables:
%%% mode: latex
%%% TeX-master: t
%%% End:
