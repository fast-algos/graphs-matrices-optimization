\documentclass[11pt]{article}
%%%%%%%%% options for the file macros.tex

\def\showauthornotes{1}
\def\showkeys{0}
\def\showdraftbox{0}
% \allowdisplaybreaks[1]


\usepackage{amsmath, ulem, amssymb, amsfonts, hyperref, amsthm, listings, tcolorbox, bbm, xifthen, soul, mathtools}
\usepackage[margin=1in]{geometry}
\usepackage[linesnumbered,lined,boxed,commentsnumbered, ruled]{algorithm2e}
\hypersetup{
    colorlinks=true,
    linkcolor=blue,
    filecolor=magenta,      
    citecolor=magenta,      
    urlcolor=cyan,
}
\usepackage[
    backend=biber,
    style=alphabetic,
    sorting=ynt,
    backref=true
]{biblatex}
\usepackage[shortlabels]{enumitem}
\usepackage{cleveref}

\newtheorem{theorem}{Theorem}[section]
\newtheorem{lemma}[theorem]{Lemma}
\newtheorem{corollary}[theorem]{Corollary}
\newtheorem{claim}[theorem]{Claim}
\newtheorem{fact}[theorem]{Fact}
\newtheorem{open-problem}[theorem]{Open Problem}

\theoremstyle{definition}

\newtheorem{definition}[theorem]{Definition}
\newtheorem{example}[theorem]{Example}

\newtheorem{remark}{Remark}
\newtheorem{question}{Question}

\newcommand{\V}[1]{\mathbf{#1}\ignorespaces}
\renewcommand\AA{\boldsymbol{\mathit{A}}}
\newcommand\LL{\boldsymbol{\mathit{L}}}
\newcommand\MM{\boldsymbol{\mathit{M}}}
\newcommand\II{\boldsymbol{\mathit{I}}}
\newcommand\JJ{\boldsymbol{\mathit{J}}}
\newcommand\KK{\boldsymbol{\mathit{K}}}

\newcommand{\ab}[1]{\langle #1 \rangle}
\renewcommand{\subset}{\subseteq}
\newcommand{\dsquare}{\mathbin{\raisebox{.2ex}{
    \hspace{-.4em}$\bigcirc$\hspace{-.75em}{\rm s}\hspace{.15em}}}}
\newcommand{\from}{\overset{R}{\leftarrow}}
\newcommand{\1}{\mathbbm{1}}

\DeclareMathOperator{\USTCON}{USTCON}
\DeclareMathOperator{\STCON}{STCON}
\allowdisplaybreaks

\usepackage{tikz}

\usepackage[
    backend=biber,
% giveninits=true,
% natbib=true,
    style=alphabetic,
    url=false, 
 %   doi=true,
    hyperref,
    backref=true,
    backrefstyle=none,
    maxbibnames=10,
    sortcites
]{biblatex}
\addbibresource{papers.bib}

%%%%%%%%% Authornotes
\newcommand{\Snote}{\Authornote{S}}

\newenvironment{tight_enumerate}{
\begin{enumerate}
 \setlength{\itemsep}{2pt}
 \setlength{\parskip}{1pt}
}{\end{enumerate}}
\newenvironment{tight_itemize}{
\begin{itemize}
 \setlength{\itemsep}{2pt}
 \setlength{\parskip}{1pt}
}{\end{itemize}}



\addbibresource{refs.bib}

%%%%%%%%%%%%%%%%%%%%%%%%%

\begin{document}

\newcommand{\coursenum}{{CSC 2421H}}
\newcommand{\coursename}{{Graphs, Matrices, and Optimization}}
\newcommand{\courseprof}{Sushant Sachdeva}

\lecturetitle{7}{Concentration Bounds}{Fengwei Sun}{29 Oct 2018}

\section{Scalar Chernoff Bound}

\begin{definition}
Let $X_1, \dots, X_t$ be independent random variables such that 

\[
	0 \le X_i \le R, \mathbb{E} \sum_{i} X_i = \sum_{i} \mathbb{E} X_i = \mu
\]

Then for all $0<\epsilon<1$, we have

\[
	P[\sum_{i} X_i \ge (1+\epsilon)\mu] \le e^{-\frac{\epsilon^2 \mu}{3R}}, P[\sum_{i} X_i \le (1-\epsilon)\mu] \le e^{-\frac{\epsilon^2 \mu}{2R}}
\]

\end{definition}

\paragraph{Example:} 

Suppose we conduct $t$ independent tosses of a fair coin. Let $X_i=\begin{cases} 1 \quad if \; heads \\ 0 \qquad o/w \end{cases}$. Then the number of heads in this trial is $\sum_{i=1}^{t} X_i$, and $\mathbb{E}(\# \; of \; heads)=\mathbb{E}\sum_{i=1}^{t} X_i = \frac{t}{2}$.\\

To obtain a good estimate of the probability that we see at least 600 heads out of 1000 tosses, we can apply the Chernoff bound with the parameters $\epsilon=0.2, R=1, \mu=500$, and get

\[
	P(at \; least \; 600 \; heads \; out \; of \; 1000 \; tosses) = P(\sum_{i=1}^{1000} X_i \ge (1+\epsilon) 500) \le e^{-\frac{0.2^2 * 500}{3}}= e^{-\frac{20}{3}} \approx e^{-7}
\]

\paragraph{Question:}

You can a coin with a bias in $\{\frac{1}{2}+\alpha, \frac{1}{2}-\alpha\}$. How many tosses do you need to decide which bias with probability of at least $1-\delta$?

\paragraph{Algorithm:}

\begin{enumerate}
	\item Toss the coin $t$ times independently;
	\item If there are at least $\frac{t}{2}$ heads, output $\frac{1}{2}+\alpha$; otherwise output $\frac{1}{2}-\alpha$.
\end{enumerate}

We would like to bound $P(failure) \le \delta$. \\

\paragraph{Case 1:}
The coin has a bias $\frac{1}{2}+\alpha$. Then $P(failure) = P(\sum X_i \le \frac{t}{2})$. \\

To apply Chernoff, we find that $R=1, \mu=t(\frac{1}{2}+\alpha)$. Since we want $(1-\epsilon)\mu = \frac{t}{2}$, we have $\epsilon = 1-\frac{t}{2\mu} = 1-\frac{1}{2\alpha}$. Therefore,
\[
	P(\sum X_i \le \frac{t}{2}) = P(\sum X_i \le (1-\epsilon)\mu) \le e^{-\frac{\epsilon^2\mu}{2}} \le \delta
\]

Since $\epsilon^2\mu \approx \Theta(t \alpha^2)$, we have 
\[
	t \ge \frac{\Theta(1)}{\alpha^2}\log \frac{1}{\delta}
\]

\paragraph{Case 2:}

The coin has a bias $\frac{1}{2}-\alpha$. Then $P(failure) = P(\sum X_i \ge \frac{t}{2})$. \\

To apply Chernoff, we find that $R=1, \mu=t(\frac{1}{2}-\alpha)$. Since we want $(1+\epsilon)\mu = \frac{t}{2}$, we have $\epsilon = \frac{t}{2\mu}-1 = \frac{1}{2\alpha}-1$. Therefore,
\[
	P(\sum X_i \ge \frac{t}{2}) = P(\sum X_i \ge (1+\epsilon)\mu) \le e^{-\frac{\epsilon^2\mu}{3}} \le \delta
\]

Since $\epsilon^2\mu \approx \Theta(t \alpha^2)$, we have 
\[
	t \ge \frac{\Theta(1)}{\alpha^2}\log \frac{1}{\delta}
\]

Both cases indicate that the $t$ is bounded by the logarithm of $\frac{1}{\delta}$, which means that $t$ would be relatively small even for very small $\delta$.

\section{Matrix Chernoff Bound}

\begin{definition}

Let $X_1, \dots, X_t \in \mathbb{R}^{d \times d}$ be symmetric independent random variables such that

\[
	0 \preceq X_i \preceq RI, \; \mu_{min} I \preceq \mathbb{E}\sum X_i \preceq \mu_{max} I
\]

Then we have

\[
	P[\lambda_{max}(\sum X_i) \ge (1+\epsilon) \mu_{max}] \le d e^{-\frac{\epsilon^2 \mu_{max}}{3R}},
\]

\[
	P[\lambda_{min}(\sum X_i) \le (1-\epsilon) \mu_{min}] \le d e^{-\frac{\epsilon^2 \mu_{min}}{2R}}
\]
	
\end{definition}

\paragraph{Note:}

\begin{enumerate}
	\item The condition that $0 \preceq X_i \preceq RI$ is equivalent to $||X_i|| \le R$, or $\lambda_{max}(X_i) \le R$, where $||A|| = \max_{x \ne 0} \frac{||Ax||}{||x||}$, and when $A$ is symmetric, $||A||=max\{\lambda_{max}, -\lambda_{min}\}$. 
	\item $\mu_{min}=\lambda_{min}(\mathbb{E}\sum X_i), \mu_{max}=\lambda_{max}(\mathbb{E}\sum X_i)$.
\end{enumerate}

\paragraph{Example:}

(Construction of random expander graphs) Suppose we would like to generate an expander graph with $n$ vertices (assuming $n$ is even).

Define a matching as a graph where $d_v=1$ for all vertices $v$. Let $H=\frac{1}{t}$(union of $t$ independent perfect matching). Notice that for all vertices $u$ in the graph $H$, $d_u = 1$. 

Using the matrix Chernoff bound, we can show that $H$ is an expander.\\

The Laplacian of $H$ is $L_H = \sum_{i} \frac{1}{t} L_i$, where $L_i$ is the Laplacian of the $i^{th}$ matching.

Let $X_i = \frac{1}{t} L_i$. We know that $X_i \succeq 0$ and $\lambda_{max}(X_i) = \frac{1}{t} \lambda_{max}(L_i)=\frac{2}{t}$.

Also, if we look at a specific vertex $u$ in each matching, it is connected to all other vertices with equal probablity $\frac{1}{n-1}$. This indicates that $\mathbb{E}L_H = \mathbb{E}L_1 = \frac{1}{n-1} L_{K_n}$. \\

Before we apply Chernoff bound on $X_i's$, one issue we notice is that $\lambda_{min}(X_i)=0$.

To fix that, we let $X_i = \frac{1}{t} L_i + \frac{1}{t(n-1)} \mathbbm{1}\mathbbm{1}^{\top}$. Now we have $\mathbb{E}\sum X_i = \frac{1}{n-1} L_{K_n} + \frac{1}{n-1} \mathbbm{1}\mathbbm{1}^{\top} = \frac{n}{n-1} I_n$, and thus $\mu_{max}=\mu_{min}=\frac{n}{n-1}$.\\

We can also show that, $\lambda_{max}(X_i) \le \frac{2}{t}$ after the change of variable.

Let $y = \hat{y} + c \frac{\mathbbm{1}}{\sqrt{n}}$ where $\hat{y}^{\top}\mathbbm{1}=0$. Then

\[
	y^{\top} X_i y = \hat{y}^{\top} (\frac{1}{t} L_i) \hat{y} + \frac{c^2n}{t(n-1)} \le (\frac{2}{t})\hat{y}^{\top}\hat{y} + \frac{n}{(n-1)t} c^2 \le \frac{2}{t} (\hat{y}^{\top}\hat{y}+c^2) \le \frac{2}{t} ||y||^2 = \frac{2}{t}
\]

\hfill \break

Now we apply the Chernoff bound, and get the following:

\[
	P[\lambda_{max}(\sum X_i) \ge (1+\epsilon) \frac{n}{n-1}] \le n e^\frac{-\epsilon^2 \frac{n}{n-1}}{3 \cdot \frac{2}{t}} = n e^{-\frac{\epsilon^2 t}{6}(\frac{n}{n-1})}
\]

If we pick $t \ge \frac{12}{\epsilon^2} \log n$, we have $P[\lambda_{max}(\sum X_i) \ge (1+\epsilon)\frac{n}{n-1}] \le n \cdot \frac{1}{n^2} = \frac{1}{n}$. \\

Similarly, we have $P[\lambda_{min}(\sum X_i) \le (1-\epsilon) \frac{n}{n-1}] \le \frac{1}{n}$. \\

Therefore, we can conclude that, with probability of at least $1 - \frac{2}{n}$,

\[
	\lambda_{max}(\sum X_i) \le (1+\epsilon)\frac{n}{n-1}, \lambda_{min}(\sum X_i) \ge (1-\epsilon)\frac{n}{n-1}
\]

or

\[
	(1-\epsilon)\frac{n}{n-1} I \preceq \sum X_i \preceq (1+\epsilon)\frac{n}{n-1}
\]

\hfill \break

To see that $H$ is a good approximation of a complete graph, let $ \Pi = I - \frac{1}{n} \mathbbm{1}\mathbbm{1}^{\top} = \frac{1}{n} L_{K_n}$. Notice that $\Pi^2 = \Pi$.

Consider $\Pi^{\top} (\sum X_i) \Pi$. We have 

\[
	(1-\epsilon)\frac{n}{n-1} \frac{1}{n} L_{K_n} \preceq \Pi^{\top} (\sum X_i) \Pi \preceq (1+\epsilon)\frac{n}{n-1} \frac{1}{n} L_{K_n}
\]

Since

\[
	\Pi^{\top} (\sum X_i) \Pi = \Pi^{\top} (L_H + \frac{1}{n-1} \mathbbm{1}\mathbbm{1}^{\top})(I - \frac{1}{n} \mathbbm{1}\mathbbm{1}^{\top}) 
\]

\[
	= \Pi^{\top} (L_H + \frac{1}{n-1} \mathbbm{1}\mathbbm{1}^{\top} - \frac{n}{n(n-1)} \mathbbm{1}\mathbbm{1}^{\top}) = \Pi^{\top} L_H = L_H
\]

Therefore,

\[
	(1-\epsilon)\frac{1}{n-1} L_{K_n} \preceq L_H \preceq (1+\epsilon)\frac{1}{n-1} L_{K_n}
\]

\hfill \break

Now we get an $\epsilon$-expander $H$ with $t \cdot \frac{n}{2} = \Theta (\frac{n\log n}{\epsilon^2})$ edges, and $L_H \approx_\epsilon L_{K_n}$.\\

\hfill \break

In general, we would like to write the Chernoff bound as the following: \\

With probability of at least $1 - 2de^{-\frac{\epsilon^2 \mu_{min}}{2R}}$, $(1-\epsilon) \mu_{min} I \preceq \sum X_i \preceq (1+\epsilon)\mu_{max} I$. \\


\begin{definition}
	
	$H=(V, E')$ is an $\epsilon$-spectral sparsifier of $G=(V, E)$ if $\frac{1}{1+\epsilon} L_G \preceq L_H \preceq (1+\epsilon) L_G$, denoted as $L_H \approx_\epsilon L_G$.
	
	Equivalently, $\forall x \in \mathbb{R}^V, \frac{1}{1+\epsilon} x^{\top} L_G x \le x^{\top} L_H x \le (1+\epsilon) x^{\top} L_G x$.
	
\end{definition}

\paragraph{Note:}

Let $x = \mathbbm{1}_S$ where $S \subset V$. Then $x^{\top} L_H x = \sum_{(u,v) \in E} w(u, v) (x(u)-x(v))^2 = |E(S, \bar{S})|$.


\paragraph{Theorem:}

For all $G=(V, E)$, there exists $H=(V, E')$ such that $L_H \approx_\epsilon L_G$ and $|E'| \le \Theta(\frac{n\log n}{\epsilon^2})$



\printbibliography
\end{document}


%%% Local Variables:
%%% mode: latex
%%% TeX-master: t
%%% End:
